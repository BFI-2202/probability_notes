\documentclass{article}
\usepackage[utf8]{inputenc}

\usepackage[T2A]{fontenc}
\usepackage[utf8]{inputenc}
\usepackage[russian]{babel}

\usepackage{tabularx}
\usepackage{amsmath}
\usepackage{pgfplots}
\usepackage{geometry}
\usepackage{multicol}
\geometry{
    left=1cm,right=1cm,top=2cm,bottom=2cm
}
\newcommand*\diff{\mathop{}\!\mathrm{d}}

\newtheorem{definition}{Определение}
\newtheorem{theorem}{Теорема}

\DeclareMathOperator{\sign}{sign}

\usepackage{hyperref}
\hypersetup{
    colorlinks, citecolor=black, filecolor=black, linkcolor=black, urlcolor=black
}

\title{Теория принятия решений}
\author{Лисид Лаконский}
\date{November 2023}

\begin{document}
\raggedright

\maketitle

\tableofcontents
\pagebreak

\section{Практическое занятие — 10.11.2023}

\subsection{Непрерывные случайные величины}

\subsubsection{Показательное (экспоненциальное) распределение}

\paragraph{Функция плотности}

$$
f(x) = \begin{cases}
    0, \ x < 0 \\
    \lambda e^{-\lambda x}, x \ge 0
\end{cases}
$$

Где $\lambda$ — параметр, являющийся положительной постоянной величиной. Характеризуется временем между независимыми событиями. Широко используется в принятии управленческих решений, прикладных дисциплинах, связанных с экономикой и надежностью работы.

\paragraph{Функция распределения}

Посчитаем для всех интервалов:

\begin{enumerate}
    \item $\int\limits_{-\infty}^{0} (0) \diff x = 0$
    \item $\int\limits_{0}^{x} (\lambda e^{-\lambda x}) = (-e^{-\lambda t}) \bigg|_{0}^{x} = 1 - e^{-\lambda x}$
    \item $0 + \int\limits_{0}^{\infty} \lambda e^{-\lambda x} \diff x = \lim\limits_{b \to +\infty} (e^{-\lambda b} + 1) = 1$
\end{enumerate}

Из этого имеем:

$$
F(x) = \begin{cases}
    0, \ x < 0 \\
    1 - e^{-\lambda x}, \ x \ge 0
\end{cases}
$$

\paragraph{Вероятность попадания случайной величины в заданный интервал}

$P(\alpha < x < \beta) = F(\beta) - F(\alpha) = (1 - e^{-\lambda \beta}) - (1 - e^{-\lambda \alpha}) = e^{-\lambda \alpha} - e^{-\lambda \beta}$ 

\paragraph{Числовые характеристики}

\begin{enumerate}
    \item \textbf{Математическое ожидание}: $M(x) = \int\limits_{0}^{+\infty} x (\lambda e^{-\lambda x}) \diff x = \begin{vmatrix}
        u = x \\
        v = -e^{-\lambda x}   
       \end{vmatrix} = \lim\limits_{b \to +\infty} = ((- x e^{-\lambda x}) \bigg|_{0}^{b} + \int\limits_{0}^{b} e^{-\lambda x} \diff x) = \lim\limits_{b \to +\infty} (- \frac{1}{\lambda} e^{-\lambda x} \bigg|_{0}^{b}) = \frac{1}{\lambda}
       $
    \item \textbf{Дисперсия}: $D(x) = M(x^2) - [M(x)]^2 = \frac{1}{\lambda^2}$
\end{enumerate}

\paragraph{Решение задач}

\paragraph{Пример №1}

$$
f(x) = \begin{cases}
    0, \ x < 0 \\
    2 e^{-2 x}, \ x \ge 0
\end{cases}
$$

$$
F(x) = \begin{cases}
    0, \ x < 0 \\
    1 - e^{-2x}
\end{cases}
$$

$P(0.4 < x < 1) = e^{-2 * 0.4} - e^{-2 * 1} \approx 0.449 - 0.135 \approx 0.314$

\textbf{Ответ:} 0.314

\paragraph{Пример №2}

Средняя длительность разговора по телефону составляет 3 минуты. Какова вероятность того, что произвольный телефонный разговор будет длиться от 3 до 6 минут.

$M(x) = 3$

$M(x) = \frac{1}{\lambda} \implies \lambda = \frac{1}{3}$

$P(3 < x < 6) = e^{-\frac{1}{3} * 3} - e^{-\frac{1}{3} * 6} = e^{-1} - e^{-2} \approx 0.232$

\textbf{Ответ:} $0.232$

\paragraph{Пример №2} Деревья в лесу растут в случайных точках, которые образуют пуассоновское поле с плотностью $\lambda$ (среднее число деревьев на ед. площади). Выбирается произвольная точка $O$ в этом лесу. Рассматриваются случайные величины: 

\begin{enumerate}
    \item $r_1$ — расстояние от точки $O$ до ближайшего дерева;
    \item $r_2$ — расстояние от точки $O$ до второго по удаленности дерева;
    \item И так далее
\end{enumerate}

\textbf{Найти закон распределения каждой из этих случайных величин}.

$$F(x) = 1 - e^{-\pi r^2 \lambda} (r > 0)$$

Плотность:

$$f(x) = \begin{cases}
    0, \ r < 0 \\
    2 \pi r \lambda e^{- \pi r^2 \lambda}, \ r \ge 0
\end{cases}$$

\begin{enumerate}
    \item $F(r_1) = 1 - e^{- \pi r_1^2 \lambda}$
    \item $F(r_2) = \dots$ (решение тривиально)
\end{enumerate}

\paragraph{Пример №3}

Пусть $\lambda = 3$,

$$f(x) = \begin{cases}
    3e^{-3x}, \ x \ge 0 \\
    0, \ x < 0
\end{cases}$$

\begin{enumerate}
    \item Построить график
    \item Найти $F(x)$

    $$F(x) = \begin{cases}
        0, \ x < 0 \\
        1 - e^{-3x}, x \ge 0
    \end{cases}$$
    \item Найти вероятность того, что случайная величина $x$ примет значение меньшее, чем ее математическое ожидание.
    
    $M(x) = \frac{1}{3}$

    $P(x < M(x)) = F(\frac{1}{3}) = 1 - e^{-3 \frac{1}{3}} = 1 - e^{-1} = 1 - \frac{1}{e} \approx 0.63$
\end{enumerate}

\subsubsection{Функция надёжности}

$R(t) = P(x > t) = 1 - F(t)$

Пусть $x$ — длительность безотказной работы, начинающейся с $t_0$, $t$ — момент, в который произошел отказ, $\lambda$ — интенсивность отказов

Определяет отказ работы прибора по показательному закону, где $F(t) = 1 - e^{-\lambda t}$

Следовательно, мы можем переписать: $R(t) = e^{-\lambda t}$

\begin{theorem}
    Вероятность безотказной работы прибора за время длительностью $t$ не зависит от того, сколько прибор работал до начала рассматриваемого промежутка времени, а зависит только от длительности $t$ и интенсивности $\lambda$.

    \textbf{Доказательство}: Пусть событие $A$ — безотказная работа прибора на интервале от $(0; t_0)$ длительностью времени $t$. Событие $B$ — безотказная работа прибора на интервале от $(t_0; t_0 + 1)$ длительностью времени $t$. Тогда безотказная работа прибора на интервале от $(t_0; t_0 + t)$ длительностью $t_0$ + $t$ — событие $AB$. вероятность которого равняется $e^{-\lambda (t_0 + t)}$

    Вычислим вероятность того, что прибор будет работать безотказно на промежутке $(t_0; t_0 + t)$ при условии, что на предшествующем интервале $(0; t_0)$ он уже проработал безотказно: $P_{A}(B) = \frac{P(AB)}{P(A)} = e^{-\lambda t}$
\end{theorem}

\paragraph{Решение задач}

\paragraph{Пример №1} Банковская компьютерная система надежности имеет показательное распределение. Какова вероятность того, что вновь установленная система проработает безотказно 100 часов?

$$F(t) = 1 - e^{-0.002 t}, \lambda = 0.002$$

$R(100) = e^{-0.002 * 100} \approx 0.82$

\textbf{Ответ:} $0.82$

\paragraph{Пример №2} Зарядки телефона хватает примерно на 90 часов. Полагая, что время, на которое хватает зарядки телефона, распределено по показательному закону, определить, какой процент телефонов будет работать без подзарядки более 100 часов. Написать функцию распределения и плотность вероятности.

$M(x) = 90$

$M(x) = \frac{1}{\lambda} \implies \lambda = \frac{1}{90}$

$R(100) = e^{-\lambda t} = e^{-\frac{100}{90}} \approx 0.33$

\subsection{Домашнее задание}

\paragraph{Задание №1} Время, в течение которого студент помнит содержание экзаменационных билетов после окончания экзамена распределяется по показательному закону с $\lambda = 0.3$. Какова доля студентов, способных вспомнить содержание экзаменационных билетов через 9 дней после экзамена?

\paragraph{Задание №2} Найти коэффициент асимметрии и эксцесс показательного распределения.

\pagebreak
\section{Практическое занятие — 11.11.2023}

\subsection{Нормальное распределение}

\subsubsection{Решение задач}

\paragraph{Пример №1}

Автомат изготавливает детали, которые отделом технического контроля считаются стандартными, если отклонение $x$ от контрольного размера не превышает $0.8$ миллиметров. Каково наиболее вероятное число стандартных деталей среди изготовленных 150, если случайная величина $x$ распределена по нормальному закону с $\sigma = 0.4$.

$P(|x - \alpha| < 0.8) = 2 \Phi_0 (\frac{0.8}{0.4}) = 2 \Phi_0(2) \approx 0.9544$

$1 - P(|x - \alpha| < 0.8) \approx 1 -0.95 = 0.05$

\textbf{Ответ}: 143

\paragraph{Пример №2}

Рост взрослого человека является случайной величиной, распределенной по нормальному закону; согласно статистике, средний рост взрослого шведа — 178 см; среднее квадратическое отклонение — 6 см. Найти вероятность того, что среди пяти встреченных шведов хотя бы один будет от 175 до 185 см.

$\Phi (\frac{185 - 178}{6}) - \Phi (\frac{175 - 178}{6}) = \Phi (\frac{7}{6}) + \Phi (\frac{1}{2}) = 0.3770 + 0.1915 = 0.5685$ — вероятность того, что шведы в пределах 175–185 см.

Вероятность того, что хотя бы один из пяти встреченных шведов будет иметь нужный рост: $1 - 0.4315^5 = 0.985$

\textbf{Ответ:} 0.985

\paragraph{Пример №3}

Длина некоторой детали представляет собой случайную величину, распределенную по нормальному закону, и имеет среднее значение 20 миллиметров. Среднее квадратическое отклонение — 0.2 мм. Найти

\begin{enumerate}
    \item $f(x) = \frac{1}{0.2 * \sqrt{2\pi}} e^{-\frac{1}{2} (\frac{x - 20}{0.2})^2} = \frac{5}{\sqrt{2 \pi}} e^{-\frac{(x - 20)^2}{0.08}}$
    \item $P(19.7 < x < 20.3) = \Phi_0 (\frac{20.3 - 20}{0.2}) - \Phi_0 (\frac{19.7 - 20}{0.2}) = 2 \Phi_0(1.5) \approx 2 * 0.4332 \approx 0.8664$
    \item $P(|x - 20| < 0.1) = 2 \Phi_0 (\frac{0.1}{0.2}) = 2 \Phi_0 (\frac{1}{2}) = 0.383$
    \item Процент от $P(|x - 20| < 0.1) = 38.3\%$
    \item Каким должно быть отклонение, чтобы процент деталей, отклонение которых от среднего не превышает заданного, повысился бы до 54\%

    $P = P(|x - 20| < \delta) = 2 \Phi_0 (\frac{\delta}{0.2}) \implies \Phi_0 (\frac{\delta}{0.2}) = 0.27 \implies \frac{\delta}{0.2} = 0.74 \implies \delta = 0.74 * 0.2 = 0.148$
    \item Интервал, симметричный относительно среднего значения такой, что $p = 0.95$
    
    $P(|x - 20| < \delta) = 0.95 = 2 \Phi_0 (\frac{\delta}{0.2}) \implies \Phi_0 (\frac{\delta}{0.2}) = 0.475 \implies \frac{\delta}{0.2} = 1.96 \implies \delta = 1.96 * 0.2 = 0.392$

    Таким образом, имеем $x \in (19.608; 20.392)$
\end{enumerate}

\subsection{Закон больших чисел}

\subsubsection{Неравенство Маркова}

Вероятность того, что случайная величина, среди которых нет отрицательных чисел, превосходит некоторое число $\epsilon > 0$ не больше, чем среднее значение, деленное на $\epsilon$:

$$P(x \ge \epsilon) \le \frac{M(x)}{\epsilon}$$

\paragraph{Пример №1}

Потребляемая организацией за сутки электроэнерегия — случайная величина с математическим ожиданием 8 КВт. Оценить вероятность того, что в ближайшие сутки потребление электроэнергии превысит 12 КВт.

$P = \frac{8}{12} \approx 0.67$

\textbf{Ответ:} $0.67$

\subsubsection{Неравенство Чебышева}

Верхняя граница вероятности события:

$$P(|x - M(x)| < \epsilon) \ge 1 - \frac{D(x)}{\epsilon^2}$$

Нижняя граница вероятности события:

$$P(|x - M(x)| \ge \epsilon) \le \frac{D(x)}{\epsilon^2}$$

\paragraph{Пример №1}

$n = 12$

$p = 0.07$

$\epsilon = 3$

$M(x) = np = 12 * 0.07 = 0.84$

$D(x) = np(1 - p) = 0.7812$

$P(|x - 0.84| < 3) \ge 1 - \frac{0.7812}{9} \approx 0.91$

\subsubsection{Теорема Чебышева}

Пусть существует $n$ независимых, либо попарно независимых, случайных величин. Обозначим их как $X_1, X_2, X_3, \dots, X_n$. При этом соблюдает условие, что их дисперсии являются ограниченными неким постоянным числом $k$. Тогда, насколько бы малым не оказалось $k$, вероятность того, что верным окажется неравенство

$$
|\frac{X_1 + X_2 + X_3 + \dots + X_n}{n} - \frac{M(X_1) + M(X_2) + M(x_3) + \dots + M(X_n)}{n}| < k
$$

Будет равняться единице.

Либо, если записать как единое выражение, будет соблюдаться

$$
\lim\limits_{n \to \infty} P(
|\frac{X_1 + X_2 + X_3 + \dots + X_n}{n} - \frac{M(X_1) + M(X_2) + M(x_3) + \dots + M(X_n)}{n}| < k) = 1
$$

Следствие:

При существовании набора случайных величин $X_1, X_2, X_3, \dots, X_n$, являющихся независимыми попарно, а также обладающих одинаковым матожиданием, обозначим его $\alpha$, при любых значениях $k > 0$, неравенство

$$
|\frac{X_1 + X_2 + X_3 + \dots + X_n}{n} - \alpha| < k
$$

Будет верным. То есть, для данной формулировки будет выполняться

$$
\lim\limits_{n \to \infty} P(|\frac{X_1 + X_2 + X_3 + \dots + X_n}{n} - \alpha| < k) = 1
$$

Кроме того можно показать, что

$$
P(|x - \alpha| < \epsilon) = 1 - \frac{\Delta}{n \epsilon^2}
$$

Смысл этой теоремы заключается в том, что среднее арифметическое достаточно большого числа независимых случайных величин с ограниченными дисперсиями утрачивает случайный характер.

\subsubsection{Теорема Бернулли}

Если в каждом из п независимых испытаний вероятность $р$ появления события $А$ постоянно, то сколь угодно близка к единице вероятность того, что отклонение относительной частоты от вероятности $р$ по абсолютной величине будет сколь угодно малым, если число испытаний $р$ достаточно велико.

$$\lim\limits_{n \to \infty} P(|\frac{m}{n} - p| < \epsilon) = 1$$

$$P(|\frac{m}{n} - p| < \epsilon) \ge 1 - \frac{p (1 - p)}{n \epsilon^2}$$

\paragraph{Пример №1}

Вероятность качественной сборки медицинского оборудования является постоянной и равняется $0.97$. Оценить вероятность того, что при сборке $500$ аппаратов относительная частота появления качественных аппаратов отклонится от вероятности этого события не более чем на $0.02$

$P(|\frac{m}{n} - 0.97| < 0.02) \ge 1 - \frac{0.0291}{0.2} \approx 0.85$

Ответ: $0.85$

\subsubsection{Теорема Пуассона}

$$\lim\limits_{n \to \infty} (|\frac{m_1 + \dots + m_n}{n} - \frac{p_1 + p_2 + \dots + p_n}{n}| < \epsilon) = 1$$

$m_{i}$ — число успехов в $i$-ом испытании

\end{document}