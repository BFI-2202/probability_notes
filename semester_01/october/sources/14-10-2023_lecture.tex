\documentclass{article}
\usepackage[utf8]{inputenc}

\usepackage[T2A]{fontenc}
\usepackage[utf8]{inputenc}
\usepackage[russian]{babel}

\usepackage{tabularx}
\usepackage{amsmath}
\usepackage{pgfplots}
\usepackage{geometry}
\geometry{
    left=1cm,right=1cm,top=2cm,bottom=2cm
}
\newcommand*\diff{\mathop{}\!\mathrm{d}}

\newtheorem{definition}{Определение}
\newtheorem{theorem}{Теорема}

\DeclareMathOperator{\sign}{sign}

\usepackage{hyperref}
\hypersetup{
    colorlinks, citecolor=black, filecolor=black, linkcolor=black, urlcolor=black
}

\title{Теория принятия решений}
\author{Лисид Лаконский}
\date{October 2023}

\begin{document}
\raggedright

\maketitle

\tableofcontents
\pagebreak

\section{Лекция — 14.10.2023}

\subsection{Асимптотические формулы}

\subsubsection{Локальная теорема Муавра–Лапласа}

Если вероятность наступления события $A$ в $n$ независимых испытаниях постоянна и равна $p$, причем $n \rightarrow \infty$, то справедливо следующее соотношение:

$$
\frac{\sqrt{n p q} P_n (k)}{\sqrt{2 n} e^{-\frac{x^2}{2}}} \rightarrow 1
$$

где $x = \frac{k - n p}{\sqrt{n p q}}$ (при $n \to \infty$)

Из этого следует:

$$
P_{n}(k) \approx \frac{1}{\sqrt{n p q}} * \frac{1}{\sqrt{2 n}} e^{-\frac{x^2}{2}}
$$

Можем записать следующим образом:

$$\phi(x) = \frac{1}{\sqrt{2 n}} e^{-\frac{x^2}{2}}$$

$$P_n(k) = \frac{1}{\sqrt{npq}} \phi(x)$$

\paragraph{Свойства}

\begin{enumerate}
    \item $\phi(x) > 0$
    \item $\phi(-x) = \phi(x)$
    \item $|x| \ge 4 \implies \phi(x) \approx 0$
\end{enumerate}

\paragraph{Пример №1}

$P_{110}(58) \approx \dots$

$x = \frac{58 - 110 * 0.4}{\sqrt{110 * 0.4 * 0.61}} = \frac{14}{5.14} \approx 2.72$

$\dots \approx \frac{1}{5.14} \phi(2.72) \approx \frac{0.0099}{5.14} \approx 0.001926$

Погрешность: $\frac{0.002025 - 0.001926}{0.002025} \approx 0.00489 \approx 5\%$

\textbf{Ответ:} $0.001926$

\paragraph{Пример №2}

$p = 0.1$, $P_{8}(3) = \dots$

$\dots = C_{8}^{3} * 0.1^3 * 0.9^5 \approx 0.033$

Теперь посчитаем по нашей новоизученной формуле:

$x = \frac{3 - 8 * 0.1}{\sqrt{8 * 0.1 * 0.9}} = \frac{2.2}{0.85} \approx 2.59$

$\dots = \frac{1}{0.85} \phi(2.59) = \frac{0.0139}{0.85} \approx 0.016$

Погрешность: $\frac{0.033 - 0.016}{0.033} \approx 0.515 \approx 51\%$

\textbf{Ответ:} 0.033

\subsubsection{Теорема Пуассона}

Если вероятность наступления события $A$ в $n$ независимых испытаниях постоянна и равна $p$, причем $n \rightarrow \infty$, $p \to 0$, $np \to \lambda$, тогда вероятность наступления $k$ успехов в $n$ испытаниях:

$$P_n(k) \to \frac{\lambda^k}{k!} e^{-\lambda}$$

где $k = np$

\textbf{Доказательство:}

$P_{n}(k) = C_{n}^{k} p^{k} q^{n - k}$

$p = \frac{\lambda}{n} \implies$

$P_{n}(k) = \frac{n!}{k!(n-k!)} (\frac{\lambda}{n})^{k} * (1 - \frac{\lambda}{n})^{n - k} = \frac{\lambda^{k}}{k!} * \frac{n - (k - 1) \dots (n - 1)}{n^{k - 1}} (1 - \frac{\lambda}{n})^{n - k} = \frac{\lambda^{k}}{k!} * \frac{n - (k - 1)}{n} \dots \frac{n - 1}{n} (1 - \frac{\lambda}{n})^{n - k} = \frac{\lambda^{k}}{n} e^{-\lambda}$

\paragraph{Пример №1}

Вероятность брака на заводе кожаных изделий составляет $p = 0.01$ Найти вероятность того, что в произведенной партии из $n = 100$ изделий не более 1 бракованного.

\textbf{Решение:}

$P = P_{100}(0) + P_{100}(1)$

$\lambda = np = 1$

$P = \frac{1^{0}}{0^{1}} e^{-1} + \frac{1}{1^1} e^{-1} = 2e^{-1} = \frac{2}{e} \approx 0.74$

\textbf{Ответ:} 0.736

\subsection{Поток событий}

\begin{definition}
    \textbf{Поток событий} — последовательность событий, которые наступают в случайные моменты времени
\end{definition}

\subsubsection{Свойства}

\begin{enumerate}
    \item Свойство стационарности: вероятность появления $k$ событий на любом промежутке времени зависит только от числа $k$ и от длительности $t$ промежутка и не зависит от начала его отсчёта.
    \item Свойство ординарности: вероятностью наступления за элементарный промежуток времени более одного события можно пренебречь по сравнению с вероятностью наступления за этот промежуток не более одного события (то есть вероятность одновременного появления двух и более событий равна нулю)
    \item Свойство отсутствия последействия: вероятность появления $k$ событий на любом промежутке времени не зависит от того, появлялись или не появлялись события в моменты времени, предшествующие началу рассматриваемого промежутка.
\end{enumerate}

\subsubsection{Виды}

\begin{definition}
    \textbf{Простейший (стационарный пуассоновский) поток} — поток событий, обладающий свойствами стационарности, ординарности и отсутствия последествия.
\end{definition}

\begin{definition}
    \textbf{Интенсивность потока} ($\lambda$) — среднее число событий, которые появляются в единицу времени.
\end{definition}

Если постоянная интенсивность потока известна, то вероятность появления $k$ событий простейшего потока за время длительностью $t$ определяется формулой Пуассона:

$$
P_{t}(k) = \frac{(\lambda t)^{k} e^{-\lambda t}}{k!}
$$

\subsection{Интегральная теорема Муавра–Лапласа}

Если вероятность наступления события $A$ в $n$ независимых испытаниях постоянна и равна $p$, то вероятность того, что в $n$ независимых испытаниях успех наступит в интервале от $k_1$ до $k_2$ при $n \to \infty$ имеет своим пределом:

$$
\frac{1}{\sqrt{2n}} \int\limits_{x_1}^{x_2} e^{-\frac{t^2}{2}} \diff t = \int\limits_{x_1}^{x_2} \phi(x) \diff x
$$


где $x_1 = \frac{k_1 - np}{\sqrt{npq}}$, $x_2 = \frac{k_2 - np}{\sqrt{npq}}$

\paragraph{Нормированная функция Лапласа}

$$\Phi_{0}(t) = \int\limits_{0}^{t} \phi^{-\frac{x^2}{2}} \diff x$$

\textbf{Свойства}:

\begin{enumerate}
    \item $\Phi_0(x) = -\Phi_0(x)$
    \item $\Phi_0(x) \uparrow \uparrow$
    \item Если $|x| \ge 5 \implies |\Phi_0(x)| = 0.5$ 
\end{enumerate}

Имеем:

$$P_{n}(k_1; k_2) = \int\limits_{x_1}^{x_2} \Phi(x) \diff x = \int\limits_{x_1}^{0} \Phi(x) \diff x + \int\limits_{0}^{x_2} \Phi(x) \diff x = -\Phi_0(x_1) + \Phi_0(x_2)$$

\paragraph{Функция Лапласа}

$$
\Phi(x) = \frac{1}{\sqrt{2n}} \int\limits_{-\infty}^{x} e^{-\frac{t^2}{2}} \diff t
$$

$$
\Phi(x) = \frac{1}{2} + \Phi_{0}^{-\infty}(x)
$$

$$
\Phi(-x) + \Phi(x) = 1
$$

Использование данной интегральной функции дает очень хорошее приближение когда $n$ большое, $p$ не маленькое, $npq > 10$

\subsection{Отклонение относительной частоты}

$P(|\frac{k}{n} - p| \le \epsilon) - ?$

$P(np - n \epsilon < k < np + n \epsilon) = \dots$

$x_1 = -\epsilon \sqrt{\frac{n}{pq}}$, $x_2 = \epsilon \sqrt{\frac{n}{pq}}$

$\dots = \frac{1}{\sqrt{2 n}} \int\limits_{-\epsilon \sqrt{\frac{n}{pq}}}^{\epsilon \sqrt{\frac{n}{pq}}} e^{-\frac{x^2}{2}} \diff x = \frac{2}{\sqrt{2n}} \int\limits_{0}^{\epsilon \sqrt{\frac{n}{pq}}} e^{-\frac{x^2}{2}} \diff x = 2 \Phi_0 (\epsilon \sqrt{\frac{n}{pq}})$

$$
P(|\frac{k}{n} - p| \le \epsilon) = 2 \Phi_0 (\epsilon \sqrt{\frac{n}{pq}})
$$


\end{document}