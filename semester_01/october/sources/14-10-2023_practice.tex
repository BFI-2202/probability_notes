\documentclass{article}
\usepackage[utf8]{inputenc}

\usepackage[T2A]{fontenc}
\usepackage[utf8]{inputenc}
\usepackage[russian]{babel}

\usepackage{tabularx}
\usepackage{amsmath}
\usepackage{pgfplots}
\usepackage{geometry}
\geometry{
    left=1cm,right=1cm,top=2cm,bottom=2cm
}
\newcommand*\diff{\mathop{}\!\mathrm{d}}

\newtheorem{definition}{Определение}
\newtheorem{theorem}{Теорема}

\DeclareMathOperator{\sign}{sign}

\usepackage{hyperref}
\hypersetup{
    colorlinks, citecolor=black, filecolor=black, linkcolor=black, urlcolor=black
}

\title{Теория вероятностей и математическая статистика}
\author{Лисид Лаконский}
\date{October 2023}

\begin{document}
\raggedright

\maketitle

\tableofcontents
\pagebreak

\section{Практическое занятие — 14.10.2023}

\subsection{Решение задач}

\paragraph{Пример №1} За 1 минуту на станцию скорой помощи поступает в среднем 3 звонка. Какова вероятность следующих событий:

\begin{enumerate}
    \item за 2 минуты поступит менее 5 звонков
    \item за 2 минуты поступит не менее 5 звонков
\end{enumerate}

\textbf{Решение:}

\begin{enumerate}
    \item $P_{2}(k < 5) = P_{2}(0) + P_2(1) + P_2(2) + P_2(3) + P_2(4) = \frac{6^0}{0!} e^{-6} + \frac{6^1}{1!} e^{-6} + \frac{6^2}{2!} e^{-6} + \frac{6^3}{3!} e^{-6} + \frac{6^4}{4!} e^{-6} = e^{-6} (1 + 6 + 18 + 36 + 54) = \frac{115}{e^6} = 0.285$
    \item $P_2(k \ge 5) = 1 - P_2(k < 5) = 1 - 0.285 = 0.715$
\end{enumerate}

\paragraph{Пример №2} Вероятность того, что товар мясокомбината не пройдет проверку на соответствие ГОСТ: $p = 0.2$. Какова вероятность того, что в партии из 400 изделий непроверенных окажется от 70 до 100?

\textbf{Решение:}

$P_400(70; 100) \approx \dots$

$x_1 = \frac{70 - 400 * 0.2}{\sqrt{400 * 0.2 * 0.8}} = \frac{-10}{8} = -1.25$

$x_2 = \frac{100 - 80}{8} = \frac{20}{8} = 2.5$

$\dots = \Phi_{0}(2.5) - \Phi_0(-1.25) = 0.4938 + 0.3944 \approx 0.8882$

\textbf{Ответ:} 0.8882

\paragraph{Пример №3}

При проведении испытаний с лазерными импульсами вероятность ошибки $p$ составляет $0.1$. Сколько испытаний нужно провести, чтобы с вероятностью $0.95$ относительная частота появления ошибки отклонилась от постоянной вероятности не более чем на $0.03$?

\textbf{Решение:}

$P(|\frac{k}{n} - 0.1| \le 0.03) = 0.95$

$2 \Phi_0 (\epsilon \sqrt{\frac{n}{pq}}) = 0.95$

$\Phi_0(0.03 \sqrt{\frac{n}{0.1*0.9}}) = 0.475$

Находим по таблице: $0.1\sqrt{n} = 1.96$

\textbf{Ответ}: $n = 384$

\paragraph{Пример №4}

Поток заявок, поступающий на телефонную станцию, представляет собой простейший поток. Количество вызовов за час составляет 30. Найти вероятность того, что за минуту поступит не менее 2 вызовов.

\textbf{Решение:}

$P = 1 - (P_{1}(0) + P_{1}(1)) = 1 - (\frac{(0.5 * 1)^{0} e^{-0.5 * 1}}{0!} + \frac{(0.5 * 1)^{1} e^{-0.5 * 1}}{1!}) = 0.0902040$

\textbf{Ответ:} $P = 0.0902040$

\paragraph{Пример №5}

При работе электронно-вычислительной машины время от времени возникают сбои. Поток сбоев можно считать простейшим. Среднее количество сбоев за сутки — 1.5. Найти вероятность следующих событий:

\begin{enumerate}
    \item За двое суток не будет ни одного сбоя
    \item В течение суток произойдет хотя бы один сбой
    \item За неделю работы машины произойдет не менее трех сбоев
\end{enumerate}

\textbf{Решение:}

\begin{enumerate}
    \item $P_2(0) = \frac{1}{1} e^{0-3} = \frac{1}{e^3} = 0.0498$
    \item $P_1(\ge 1) = 1 - P_1(0) = 1 - e^{-1.5} = 1 - \frac{1}{e^{1.5}} = 0.777$
    \item $P_7(\ge 3) = 1 - P_7(0) - P_7(1) - P_7(2) = 1 - e^{-10.5} (1 + \frac{10.5}{1!} + \frac{10.5^2}{2!}) = 0.998$
\end{enumerate}

\paragraph{Пример №6}

На перекрестке стоит автоматический светофор, в котором 1 минуту горит зеленый свет и 0.5 минуты красный; затем опять 1 минуту горит зеленый свет и 0.5 минуты красный; и так далее. В какой-то случайный момент времени подъезжает автомобиль. Найти вероятность того, что он проедет его, не останавливаясь. Каково среднее время ожидания?

\textbf{Решение:}

Гипотезы:

\begin{enumerate}
    \item $H_1$ — приехал на красный
    \item $H_2$ — приехал на зелёный
\end{enumerate}

Вероятности наступления гипотез:

\begin{enumerate}
    \item $P(H_1) = \frac{1}{3}$
    \item $P(H_2) = \frac{2}{3}$
\end{enumerate}

$T = \frac{2}{3} * 0 + \frac{1}{3} * \frac{1}{4} = \frac{1}{12}$

\textbf{Ответ:} $\frac{2}{3}; \frac{1}{12}$

\paragraph{Пример №7}

Число атак истребителей, которым может подвергнуться бомбардировщик над территорией противника, есть случайная величина, распределенная по закону Пуассона с $\lambda = 3$. Каждая атака с вероятностью $0.4$ заканчивается поражением бомбардировщика. Определить

\begin{enumerate}
    \item Вероятность поражения бомбардировщика
    \item Ту же вероятность, если количество атак ровно $3$
\end{enumerate}

\textbf{Решение:}

\begin{enumerate}
    \item $\lambda = np = 3 * 0.4 = 1.2$
    
    $P = 1 - P_3(1) = 1 - 0.361 = 0.639$ 
    \item $P = 1 - 0.6^3 = 0.784$ 
\end{enumerate}

\paragraph{Пример №8}

Производится стрельба тремя независимыми выстрелами по цели, имеющей вид полосы. Ширина полосы $20$ метров. Прицеливание производится по средней линии полосы. Систематическая ошибка отсутствует. Среднее квадратическое отклонение — 16 метров. Найти вероятность попадания в полосу при одном выстреле.

\textbf{Решение:}

$P = 2 \Phi (\frac{a}{\sigma}) - 1 = \dots$

$\dots = 2 * \Phi (0.63) - 1 = 2 * (\frac{1}{2} + 0.2357) - 1 = 0.4714$

\textbf{Ответ:} 0.4714

\subsection{Домашнее задание}

Решить задачи с.56, №5–№11 (простое можно не решать)

\end{document}