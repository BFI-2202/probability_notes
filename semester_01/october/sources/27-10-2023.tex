\documentclass{article}
\usepackage[utf8]{inputenc}

\usepackage[T2A]{fontenc}
\usepackage[utf8]{inputenc}
\usepackage[russian]{babel}

\usepackage{tabularx}
\usepackage{amsmath}
\usepackage{pgfplots}
\usepackage{geometry}
\usepackage{multicol}
\geometry{
    left=1cm,right=1cm,top=2cm,bottom=2cm
}
\newcommand*\diff{\mathop{}\!\mathrm{d}}

\newtheorem{definition}{Определение}
\newtheorem{theorem}{Теорема}

\DeclareMathOperator{\sign}{sign}

\usepackage{hyperref}
\hypersetup{
    colorlinks, citecolor=black, filecolor=black, linkcolor=black, urlcolor=black
}

\title{Теория принятия решений}
\author{Лисид Лаконский}
\date{October 2023}

\begin{document}
\raggedright

\maketitle

\tableofcontents
\pagebreak

\section{Практическое занятие — 27.10.2023}

\subsection{Случайные величины}

Случайные величины разделяются на

\begin{enumerate}
    \item \textbf{Дискретные} — конечное количество; например, цифры на кубике.
    \item \textbf{Непрерывные} — те случайные величины, которые разделяют сплошь промежуток, их может быть бесконечно много.
\end{enumerate}

\subsubsection{Дискретные случайные величины (ДСВ)}

\textbf{Законом распределения дискретной случайной величины} называют соответствие между полученными значениями дискретной случайной величины и их вероятностями.

Его можно задать:

\begin{enumerate}
    \item таблично;
    \item графически;
    \item аналитически (в виде формулы).
\end{enumerate}

\textbf{Ограничение}:

$$
\sum\limits_{i = 1}^{n} p_{i} = 1
$$

\textbf{Функция распределения} обозначается следующим образом: $F_{X}(x)$, $F(x)$ — вероятность того, что случайная величина примет значение \textbf{строго меньше} $x$.

\textbf{Свойства} функции распределения:

\begin{multicols}{2}
\begin{enumerate}
    \item $0 \le F(x) \le 1$
    \item $\lim\limits_{x \to -\infty} F(x) = 0$
    \item $\lim\limits_{x \to +\infty} F(x) = 1$
    \item $F(x_2) \ge F(x_1)$, если $x_2 \ge x_1$
\end{enumerate}
\end{multicols}

\paragraph{Числовые характеристики:}

\begin{enumerate}
    \item \textbf{Математическое ожидание}: $M(x)$, $m_x$, $E(x)$ — характеризует среднее арифметическое.
    
    $$M(x) = x_1p_1 + x_2 p_2 + \dots + x_{n} p_{n} = \sum\limits_{i = 1}^{n} x_{i} p_{i}$$

    $$M(x^2) = \sum\limits_{i = 1}^{n} x_{i}^2 p_{i}$$

    \begin{enumerate}
        \item Матожидание от любой константы равно константе: $M(c) = c$
        \item Константы выносятся за знак матожидания: $M(cx) = c M(x)$
        \item Матожидание суммы равно сумме матожиданий: $M(x + y) = M(x) + M(y)$
        \item Матожидание произведения равно произведению матожиданий: $M(x * y) = M(x) * M(y)$
    \end{enumerate}

    \textbf{Отклонение} от математического ожидания:

    $$X - M(x)$$
    \item \textbf{Дисперсия}: $D(x)$ — матожидание от отклонения в квадрате.
    
    $$D(x) = M((x - M(x))^2)$$

    \begin{multicols}{2}
    \begin{enumerate}
        \item $D(c) = 0$
        \item $D(cx) = c^2 D(x)$
        \item $D(x + y) = D(x) + D(y)$
        \item $D(x - y) = D(x) + D(y)$
    \end{enumerate}
    \end{multicols}
    \item \textbf{Среднеквадратическое отклонение}:
    
    $\sigma(x) = \sqrt{D(x)}$
\end{enumerate}

\begin{theorem}
    Математическое ожидание независимых случайных величин в $n$ испытаниях при постоянной и одинаковой вероятности успеха в каждом испытании $p$:

    $$M(x) = np$$
\end{theorem}

\begin{theorem}
    Матожидание отклонения случайной величины от математического ожидания равно нулю:

    $$M(x - M(x)) = 0$$

    \textbf{Доказательство}:

    $M(X - M(x)) = M(x) - M(M(x)) = M(x) - M(x) = 0$
\end{theorem}

\begin{theorem}
    Дисперсия в $n$ независимых испытаниях при постоянной и одинаковой вероятности успеха в каждом испытании $p$:

    $$D(x) = np(1-p)$$
\end{theorem}

\begin{theorem}
    Дисперсия случайной величины $x$:
    $$D(x) = M(x^2) - M(x)^2$$

    \textbf{Доказательство}:

    $D(x) = M(x - M(x))^2 = M(x^2 - 2xM(x) + M(x)^2) = M(x^2) 2 M(x) M(M(x)) + M(M(x)^2) = M(x^2) - M(x)^2$
\end{theorem}

\begin{definition}
    Случайная величина $x_0$ называется \textbf{нормированной}, если
    
    \begin{enumerate}
        \item $M(x_0) = 0$
        \item $\sigma(x_0) = 1$
    \end{enumerate}

    Если $M(x) = a$, $\sigma(x) = \sigma$, то $$x_{0} = \frac{x - a}{\sigma}$$

    Доказательство:
    
    \begin{enumerate}
        \item $M(x_0) = \frac{1}{\sigma} M(x - a) = \frac{1}{\sigma} (M(x) - M(a)) = \frac{1}{\sigma} (a - a) = 0$
        \item $D(x_0) = \frac{1}{\sigma^2} D(x - a) = \frac{1}{\sigma^2} (D(x) + D(a)) = \frac{1}{\sigma^2} * \sigma^2 = 1$
    \end{enumerate}
\end{definition}

\begin{definition}
    \textbf{Модой дискретной случайной величины} $x$ называют значение, которое принимается с наибольшей вероятностью. Обозначение: $m_o x$

    $$max \ P(x = x_i) = P(x = m_o x)$$
\end{definition}

\begin{definition}
    \textbf{Медианой дискретной случайной величины} $x$ является такое число, для которого вероятность меньших значений меньше $0.5$ и вероятность больших значений меньше $0.5$. Обозначение: $m_{e} x$
\end{definition}


\paragraph{Пример №1}

Закон распределения для шестигранного кубика.

$$
\begin{pmatrix}
  1 & 2 & 3 & 4 & 5 & 6 \\
  \frac{1}{6} & \frac{1}{6} & \frac{1}{6} & \frac{1}{6} & \frac{1}{6} & \frac{1}{6}
\end{pmatrix}
$$

\textbf{Функция распределения}:

$$F(x) = \begin{cases}
    0, \ x \le 1 \\
    \frac{1}{6}, 1 < x \le 2 \\
    \frac{2}{6}, 2 < x \le 3 \\
    \dots \\
    1, 6 < x
\end{cases}
$$

По ней мы можем нарисовать простенький график, что оставляется в качестве упражнения читателю.

Также найдем \textbf{математическое ожидание}: $M(x) = \frac{1}{6} (1 + 2 + 3 + 4 + 5 + 6) = \frac{21}{6}$

$D(x) = \frac{1}{6} ((-2.5)^2 + (-1.5)^2 + (-0.5)^2 + (0.5)^2 + (1.5)^2 + (2.5)^2) = \frac{35}{12} \approx 2.92$

\paragraph{Пример №2}

Найдем математическое ожидание для следующих величин:

$$
\begin{pmatrix}
    x & 0.01 & -0.01 \\
    p & \frac{1}{2} & \frac{1}{2}
\end{pmatrix}
$$


$$
\begin{pmatrix}
    y & 100 & -100 \\
    p & \frac{1}{2} & \frac{1}{2}
\end{pmatrix}
$$

Матожидание для обеих этих величин — ноль.

\paragraph{Пример №3}

Производится испытание на надежность трех видеорегистраторов. Вероятность то, что видеорегистратор выйдет из строя: $0.15$. Найти математическое ожидание и дисперсию.

\textbf{Закон распределения}:

$$
\begin{pmatrix}
    x & 0 & 1 & 2 & 3 \\
    p & 0.15^3 & & & 0.85^3
\end{pmatrix}
$$

$P_3(1) = C_{3}^{1} 0.15^2 * 0.85 \approx 0.0574$

$P_3(2) = 3 * 0.15 * 0.85^2 \approx 0.3251$

$$
\begin{pmatrix}
    x & 0 & 1 & 2 & 3 \\
    p & 0.0034 & 0.0574 & 0.3251 & 0.6141
\end{pmatrix}
$$

\textbf{Математическое ожидание}: $$M(x) = 1 * 0.0574 + 2 * 0.3251 + 3 * 0.6141 \approx 2.55$$

\textbf{Дисперсия}:

$D(x) = M(x^2) - (2.55)^2 = 0.3825$

Могли бы посчитать математическое ожидание и дисперсию с помощью формул:

$$
M(x) = 3 * 0.85 \approx 2.55
$$

$$
D(x) = 3 * 0.85 * 0.15 \approx 0.3825
$$

\paragraph{Пример №4}

$$
\begin{pmatrix}
    x & 0 & 1 \\
    p & 0.8 & 0.2
\end{pmatrix}
$$

\textbf{Найти}:

\begin{enumerate}
    \item $m_{o} x = 0$;
    \item $m_{e} x$ не существует.
\end{enumerate}

\end{document}