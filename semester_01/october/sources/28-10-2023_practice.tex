\documentclass{article}
\usepackage[utf8]{inputenc}

\usepackage[T2A]{fontenc}
\usepackage[utf8]{inputenc}
\usepackage[russian]{babel}

\usepackage{tabularx}
\usepackage{amsmath}
\usepackage{pgfplots}
\usepackage{geometry}
\usepackage{multicol}
\geometry{
    left=1cm,right=1cm,top=2cm,bottom=2cm
}
\newcommand*\diff{\mathop{}\!\mathrm{d}}

\newtheorem{definition}{Определение}
\newtheorem{theorem}{Теорема}

\DeclareMathOperator{\sign}{sign}

\usepackage{hyperref}
\hypersetup{
    colorlinks, citecolor=black, filecolor=black, linkcolor=black, urlcolor=black
}

\title{Теория принятия решений}
\author{Лисид Лаконский}
\date{October 2023}

\begin{document}
\raggedright

\maketitle

\tableofcontents
\pagebreak

\section{Практическое занятие — 28.10.2023}

\subsection{Непрерывные случайные величины}

\subsubsection{Решение задач}

\paragraph{Пример №1} В стране пять крупных автогигантов. В кризисных условиях риск того, что завод обанкротится, составляет 30\%. Написать закон распределения автогигантов, которые могут обанкротиться в предстоящий кризисный период. Найти математическое ожидание и дисперсию.

Используем биномиальное распределение, $x$ — количество возможно обанкротившихся заводов:

$$
\begin{pmatrix}
  x & 0 & 1 & 2 & 3 & 4 & 5 \\
  p & 0.7^5 & 5*0.3*0.7^4 & 10*0.3^2*0.7^3 & 10*0.3^3*0.7^2 & 10*0.3^4*0.7 & 0.3^5  \\
    & 0.168 & 0.36 & 0.31 & 0.132 & 0.028 & 0.002
\end{pmatrix}
$$

\begin{enumerate}
    \item $M(x) = 5 * 0.3 = 1.5$
    \item $D(x) = 1.05$
\end{enumerate}

Таким образом, ожидать можно, что штуки две автогигантов рухнут, а чтобы все рухнули — можно не ожидать, \textbf{всё нормально}.

\paragraph{Пример №2} Количество ДТП в городе за неделю является случайной величиной, распределенной согласно Пуассоновскому распределению со средним значением, равным трём. \textbf{Какова вероятность того, что случится меньше трех ДТП}?

$M(x) = \lambda = 3$

$P(x < 3) = P(0) + P(1) + P(2) = e^{-3} (1 + 3 + \frac{9}{2}) \approx 0.42$

\textbf{Ответ:} 0.42

\paragraph{Пример №3}

Пусть $$F(x) = \begin{cases}
    0, \ x \le 1 \\
    (x - 1)^2, \ 1 < x \le 2 \\
    1, \ x > 2
\end{cases}
$$

Построить функцию распределения и найти вероятность того, что СВ примет значение

\begin{multicols}{2}
    \begin{enumerate}
        \item Принадлежащее $(1.2; 1.6)$
        \item Принадлежашее $[1.7; 2.3]$
        \item Принадлежащее $\{ x: x > 1.5 \}$
        \item Принадлежащее $\{ x, \ x \le 1.3 \}$
    \end{enumerate}    
\end{multicols}

Решение:

\begin{enumerate}
    \item $P(1.2 < x < 1.6) = F(1.6) - F(1.2) = 0.32$
    \item \dots
    \item $P(1.5 < x < +\infty) = 1 - F(1.5) = 1 - 0.25 = 0.75$
    \item $P(1.3 < x < +\infty) = F(1.3) = 0.09$
\end{enumerate}

\paragraph{Пример №4}

Пусть $f(x) = \frac{c}{1 + 9x^2}$. Найти значение $c$.

$\int\limits_{-\infty}^{+\infty} \frac{c}{1 + 9x^2} \diff x = 1$

$\int\limits_{-\infty}^{+\infty} \frac{c}{1 + 9x^2} = \lim\limits_{a \to -\infty, b \to +\infty} \int\limits_{a}^{b} \frac{c \diff x}{1 + 9x^2} = \lim\limits_{a \to -\infty, b \to +\infty} \int\limits_{a}^{b} \frac{c}{3} \arctg 3x \bigg|_{a}^{b} = \lim\limits_{a \to -\infty, b \to +\infty} \frac{c}{3} (\arctg 3b - \arctg 3a) = \frac{c}{3} (\frac{\pi}{2} + \frac{\pi}{2}) = \frac{\pi c}{3} = 1 \implies c = \frac{3}{\pi}$

\textbf{Ответ}: $\frac{3}{\pi}$

\paragraph{Пример №5}

Пусть $$f(x) = \begin{cases}
    0, \ x \le 2 \\
    2x - 4, \ 2 < x \le 3 \\
    0, \ x > 3
\end{cases}$$

Найти $F(x)$:

\begin{enumerate}
    \item Если $x \in (-\infty; 2]$
    
    $F(x) = \int\limits_{-\infty}^{2} 0 \diff t = 0$
    \item Если $x \in (2; 3]$
    
    $F(x) = \int\limits_{2}^{x} (2x - 4) \diff x = (x^2 - 4x) \bigg|_{2}^{x} = x^2 - 4x + 4 = (x-2)^2$
    \item Если $x \in (3; +\infty)$
    
    $F(x) = 1$
\end{enumerate}

\paragraph{Пример №6}

Пусть $$f(x) = \begin{cases}
    0, x \le 0 \\
    2e^{-2x}, \ x > 0 
\end{cases}$$

Определить \textbf{математическое ожидание}. $M(x) = \int\limits_{-\infty}^{+\infty} x f(x) \diff x = \int\limits_{-\infty}^{0} x 0 \diff x + \int\limits_{0}^{+\infty} x 2 e^{-2x} \diff x = \int\limits_{0}^{+\infty} x e^{-2x} \diff x = \lim\limits_{b \to +\infty} 2 \int\limits_{0}^{b} x e^{-2x} \diff x = \begin{vmatrix}
    u = x & \diff u = \diff x \\
    \diff v = 2 e^{-2} \diff x & v = -e^{-2x}
\end{vmatrix} = \lim\limits_{b \to +\infty} (-x e^{- 2x} \bigg|_0^b + \int\limits_{0}^{b} e^{-2x} \diff x) = \lim\limits_{b \to +\infty} (-b e^{- 2 b} - \frac{1}{2} e^{-2x} \bigg|_{0}^{b}) = \lim\limits_{b \to +\infty} (-b e^{-2 b} - \frac{1}{2} e^{-2b} + \frac{1}{2}) = 0.5$

\paragraph{Пример №7}

Пусть $$f(x) = \begin{cases}
    0, \ x \le 1 \\
    -6x^2 + 18x - 12, \ 1 < x < 2 \\
    0, x \ge 2
\end{cases}$$

\textbf{Найти}: $M_e x$

Для того, чтобы найти $M_e x$, необходимо найти $F(x)$.

$\int\limits_{1}^{x} (-6x^2 + 18x - 12) \diff x = -2x^3 + 9x^2 - 12 x \bigg|_1^x = -2x^3 + 9x^2 - 12x + 5$

$$F(x) = \begin{cases}
    0, \ x \le 1 \\
    -2x^3 + 9x^2 - 12x + 5, \ 1 < x < 2 \\
    1, \ x \ge 2
\end{cases}$$

$$-2x^3 + 9x^2 - 12x + 5 = \frac{1}{2}$$

$$x_1 = \frac{3}{2}, x_2 = \frac{3}{2} - \frac{\sqrt{3}}{2}, x_3 = \frac{3}{2} + \frac{\sqrt{3}}{2}$$

\textbf{Ответ}: $\frac{3}{2}$

\subsection{Равномерное распределение}

Говорят, что СВ имеет непрерывное равномерное распределение на отрезке $[ a , b ]$ $[a,b]$, где $a,b \in R$, если её плотность $f_{X}(x)$ имеет вид: 

$$f(x) = \begin{cases}
    c, \ x \in (a, b) \\
    0, \ x \notin (a, b)
\end{cases} = \begin{cases}
    \frac{1}{b - a}, \ x \in (a, b) \\
    0, \ x \notin (a, b)
\end{cases}$$

\subsubsection{Функция распределения}

Интегрируя определённую выше плотность, получаем: 

$$F_x(x) = P(X \le x) = \begin{cases}
    0, \ x < a \\
    \frac{x - a}{b - a}, \ a \le x < b \\
    1, \ x \ge b
\end{cases}$$

\subsubsection{Числовые характеристики}

\begin{multicols}{2}
    \begin{enumerate}
        \item \textbf{Математическое ожидание:} $M(x) = \frac{a + b}{2}$
        \item \textbf{Дисперсия}: $D(x) = \frac{(b - a)^2}{12}$
        \item \textbf{Значение медианы}: $\frac{a + b}{2}$
        \item \textbf{Значение моды}: любое число из $[a, b]$
        \item \textbf{Коэффициент асимметрии}: $0$
        \item \textbf{Эксцесс}: $-\frac{6}{5}$
    \end{enumerate}    
\end{multicols}

\subsubsection{Решение задач}

\paragraph{Пример №1} В некотором городе трамвай ходит по расписанию с интервалом 10 минут. Найти вероятность того, что пассажир, подошедший к остановке, будет ожидать трамвай более 2 минут.

\textbf{Ответ:} $P = 0.8$

\paragraph{Пример №2} Обследуются животные, каждое из которых с вероятностью $p$ является больным. Обследование производится путем анализов крови. Если смешать кровь $n$ животных, то анализ этой смеси будет положительным, если среди этих $n$ животных хотя бы одно больное. Требуется обследовать большое число животных $N$. Предлагается несколько способов обследования:

\begin{enumerate}
    \item \textbf{Обследовать всех этих животных}, проведя $N$ анализов;
    \item \textbf{Вести обследование по группам}, смешав сначала кровь группы из $n$ животных, если все хорошо — перейти к следующей группе, если нет — сделать анализ каждого из животных.
\end{enumerate}

\textbf{Определить, какой способ обследования наиболее выгодный} (в плане минимального проведения анализов): первый или второй?

\textbf{Закон распределения для группы животных}:

$$
\begin{pmatrix}
    x & 1 & n + 1 \\
    p & (1-p)^{n} & 1 - (1-p)^{n} 
\end{pmatrix}
$$

$M(x_1) = n$

$M(x_2) = (1-p)^{n} + (n + 1)(1 - (1-p)^{n}) = (1-p)^{n} + n - n(1-p)^{n} + 1 - (1-p)^{n} = n - n(1-p)^{n} + 1$

Предполагаем, второй способ выгоднее: $n - n(1-p)^{n} + 1 > n \Longleftrightarrow n(1-p)^{n} < 1$ — \textbf{если это значение меньше единице, используем второй способ; если больше — первый способ}.

\subsection{Домашнее задание}

\paragraph{Задание №1}

Пусть $$F(x) = \begin{cases}
    0, \ x \le 1 \\
    (x - 1)^2, \ 1 < x \le 2 \\
    1, \ x > 2
\end{cases}
$$

Построить функцию распределения и найти вероятность того, что СВ примет значение

\begin{multicols}{2}
    \begin{enumerate}
        \item Принадлежашее $[1.7; 2.3]$
    \end{enumerate}    
\end{multicols}

\paragraph{Задание №2}

Пусть

$$f(x) = \begin{cases}
    0, \ x \le 2 \\
    (x-2)^2, \ 2 < x \le 3 \\
    1, \ x > 3
\end{cases}$$

Найти $F(x)$

\end{document}