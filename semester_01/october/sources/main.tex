\documentclass{article}
\usepackage[utf8]{inputenc}

\usepackage[T2A]{fontenc}
\usepackage[utf8]{inputenc}
\usepackage[russian]{babel}

\usepackage{tabularx}
\usepackage{amsmath}
\usepackage{pgfplots}
\usepackage{geometry}
\geometry{
    left=1cm,right=1cm,top=2cm,bottom=2cm
}
\newcommand*\diff{\mathop{}\!\mathrm{d}}

\newtheorem{definition}{Определение}
\newtheorem{theorem}{Теорема}

\DeclareMathOperator{\sign}{sign}

\usepackage{hyperref}
\hypersetup{
    colorlinks, citecolor=black, filecolor=black, linkcolor=black, urlcolor=black
}

\title{Теория принятия решений}
\author{Лисид Лаконский}
\date{October 2023}

\begin{document}
\raggedright

\maketitle

\tableofcontents
\pagebreak

\section{Практическое занятие — 13.10.2023}

$P_{8}{5} = C_{8}^{5} (\frac{1}{2})^8 = 7 * 2^3 * \frac{1}{2^{8}} = \frac{7}{32}$ — вероятность выиграть пять партий из восьми у равносильного противника

$P_{4}{3} = C_{4}^{3} (\frac{1}{2})^4 = 4 * \frac{1}{2^{4}} = \frac{8}{32}$ — вероятность выиграть четыре партии из восьми у равносильного противника

\subsection{Решение задач}

\paragraph{Задача №1}

Человек, принадлежащий к определенной группе населения, с вероятность 0.2 оказывается брюнетом, 0.3 — шатеном, 0.4 — блондином, 0.1 — рыжим. Выбирается группа из шести человек. Найти вероятности следующих событий:

\begin{enumerate}
    \item Событие $A$ — хотя бы один рыжий
    \item Событие $B$ — в составе группы не меньше четырех блондинов
    \item Событие $C$ — в составе группы равное число блондинов и шатенов
\end{enumerate}

\textbf{Решение:}

\begin{enumerate}
    \item Вероятность того, что есть хотя бы один рыжий:

    $P(A) = 1 - 0.9^6 \approx 0.468$
    \item Вероятность того, что не меньше четырех блондинов:

    $P(B) = C_{6}^{4} 0.4^{4} 0.6^{6 - 4} + C_{6}^{5} 0.4^{5} 0.6^{6 - 5} + C_{6}^{6} 0.4^{6} 0.6^{6 - 6} = 0.1792$
    \item Вероятность того, что равное число блондинов и шатенов:
    
    $P(C) = P(C_{00}) + P(C_{11}) + P(C_{22}) + P(C_{33})$

    $P(C_{00}) = (1-0.7)^{6} \approx 0.0007$

    Вероятность того, что будет один блондин, один шатен и четыре остальных:

    $P(C_{11}) = P_6(1; 1; 4) = \frac{6!}{1!1!4!} * 0.3 * 0.4 * (1 - 0.7)^4 \approx 0.03$

    Вероятность того, что будет два блондина, два шатена и два остальных:

    $P(C_{22}) = P_6(2; 2; 2) = \frac{6!}{2!2!2!} * 0.3 * 0.4 * (1 - 0.7)^2 \approx 0.1215$

    Вероятность того, что будет три блондина, три шатена и ноль остальных:

    $P(C_{33}) = P_6(3; 3) = \frac{6!}{3!3!} * 0.3 * 0.4 \approx 0.03$

    $P(C) = 0.0007 + 0.03 + 0.1215 + 0.03 \approx 0.18$
\end{enumerate}

\paragraph{Задача №2}

Производится стрельба по цели тремя снарядами. Снаряды попадают в цель независимо друг от друга. Для каждого снаряда вероятность попадания в цель $0.4$. Если в цель попал один снаряд, он ее поражает с вероятностью $0.3$. Если в цель попало два снаряда — с вероятностью $0.7$. Если в цель попало три снаряда — с вероятностью $0.9$. \textbf{Найти полную вероятность поражения цели}.

Гипотезы:

\begin{enumerate}
    \item $H_{1}$ — попал один снаряд
    \item $H_{2}$ — попало два снаряда
    \item $H_{3}$ — попало три снаряда
\end{enumerate}

Найдем вероятности для каждой из них:

\begin{enumerate}
    \item $P(H_{1}) = C_{3}^{1} * 0.4 * 0.6^{2} = 0.432$
    \item $P(H_{2}) = C_{3}^{2} * 0.4^{2} * 0.6 = 0.288$
    \item $P(H_{3}) = 0.4^3 = 0.064$
\end{enumerate}

\textbf{Ответ}:

$P(A) = 0.432 * 0.3 + 0.288 * 0.7 + 0.064 * 0.9 = 0.3888$

\paragraph{Задача №3}

Имеется $n$ лунок, по которым случайным образом разбрасывается $m$ шариков. Найти вероятность того, что в первую лунку попадет ровно $k$ шариков.

\textbf{Ответ:}

$P(A) = C_{m}^{k} * (\frac{1}{n})^{k} * (1 - \frac{1}{n})^{m - k}$

\paragraph{Задача №4}

Мишень состоит из яблока и двух колец. При одном выстреле вероятность попадания в яблоко равняется 0.4, вероятность попадания в первое кольцо — 0.5, вероятность попадания во второе кольцо — 0.6, вероятность непопадания — 0.3.

По мишени производится пять выстрелов. \textbf{Найти вероятность того, что они дадут два попадания в яблоко и одно попадание во второе кольцо}.

\textbf{Ответ:}

$P(A) = P_{5}(2;0;1;2) = \frac{5!}{2!*0!*1!*2!} * 0.4^2 * 0.5^0 * 0.6^1 * 0.3^2 = 0.2592$

\paragraph{Задача №5}

Производится стрельба пятью снарядами по группе, состоящей из трех целей. Обстрел ведется в следующем порядке: сначала обстреливается первая цель и огонь по ней ведется до тех пор, пока она или не будет поражена, или не кончатся все пять снарядов. После поражения обстрел переходит на следующую цель и так далее. Вероятность поражения цели при стрельбе по ней одним выстрелом — $p$

Найти вероятности того, что будет поражено:

\begin{enumerate}
    \item 0 целей
    \item 1 цель
    \item 2 цели
    \item 3 цели
\end{enumerate}

Решение:

\begin{enumerate}
    \item $P_0 = (1 - p)^5$
    \item $P_1 = P_{5}(1) = C_{5}^{1} p (1 - p)^4$
    \item $P_2 = P_{5}(2) = C_{5}^{2} p^2 (1 - p)^3$
    \item $P_3 = P_{5}(3) = C_{5}^{3} p^3 (1 - p)^2$
\end{enumerate}

\paragraph{Задача №6}

Прибор, состоящий из четырех узлов, работал в течение времени $t$. Надежность (вероятность безотказной работы) каждого узла за это время $t$ равна $0.7$. По истечении времени $t$ прибор останавливается, техник осматривает его и заменяет узлы, вышедшие из строя. На замену одного узла ему требуется время $\tau$. Найти вероятность того, что через время $2 \tau$ после остановки прибор будет готов для нормальной работы.

\textbf{Ответ:}

$P = P_{4}^{0} + P_{4}^{1} + P_{4}^{2} = C_{4}^{0} * 0.3^{0} * 0.7^{4} + C_{4}^{1} * 0.3^{1} * 0.7^{3} + C_{4}^{2} * 0.3^{2} * 0.7^{2} = 0.9163$

\paragraph{Задача №7}

В урне имеется $k$ шаров. Каждый из них с вероятностью $\frac{1}{2}$ может оказаться белым или черным. Из урны вынимается $n$ раз по одному шару. Причем каждый вынутый шар каждый раз возвращается обратно и шары перемешиваются. Среди вынутых $n$ шаров $m$ шаров оказались белыми. \textbf{Найти вероятность того, что из $k$ шаров в урне ровно $l$ белых}.

\textbf{Формула Байеса:}

$P(A|B) = \frac{P(B|A) P(A)}{P(B)}$

где

\begin{enumerate}
    \item $P(A)$ — априорная вероятность гипотезы $A$
    \item $P(A|B)$ — вероятность гипотезы $A$ при наступлении события $B$
    \item $P(B|A)$ — вероятность наступления события $B$ при истинности гипотезы $A$
    \item $P(B)$ — полная вероятность наступления события $B$
\end{enumerate}

\textbf{Решение:}

$P(H_{l}) = C_{k}^{l} (\frac{1}{2})^{k}$

$P(A|H_{l}) = C_{n}^{m} (\frac{l}{k})^{m} (1 - \frac{l}{k})^{n - m}$

$P(H_{l}|A) = \frac{P(A|H_{l}) P(H_{l})}{P(A)}$

\end{document}