\documentclass{article}
\usepackage[utf8]{inputenc}

\usepackage[T2A]{fontenc}
\usepackage[utf8]{inputenc}
\usepackage[russian]{babel}

\usepackage{tabularx}
\usepackage{amsmath}
\usepackage{pgfplots}
\usepackage{geometry}
\usepackage{multicol}
\geometry{
    left=1cm,right=1cm,top=2cm,bottom=2cm
}
\newcommand*\diff{\mathop{}\!\mathrm{d}}

\newtheorem{definition}{Определение}
\newtheorem{theorem}{Теорема}

\DeclareMathOperator{\sign}{sign}

\usepackage{hyperref}
\hypersetup{
    colorlinks, citecolor=black, filecolor=black, linkcolor=black, urlcolor=black
}

\title{Теория вероятностей и математическая статистика}
\author{Лисид Лаконский}
\date{October 2023}

\begin{document}
\raggedright

\maketitle

\tableofcontents
\pagebreak

\section{Практическое занятие — 13.10.2023}

$P_{8}{5} = C_{8}^{5} (\frac{1}{2})^8 = 7 * 2^3 * \frac{1}{2^{8}} = \frac{7}{32}$ — вероятность выиграть пять партий из восьми у равносильного противника

$P_{4}{3} = C_{4}^{3} (\frac{1}{2})^4 = 4 * \frac{1}{2^{4}} = \frac{8}{32}$ — вероятность выиграть четыре партии из восьми у равносильного противника

\subsection{Решение задач}

\paragraph{Задача №1}

Человек, принадлежащий к определенной группе населения, с вероятность 0.2 оказывается брюнетом, 0.3 — шатеном, 0.4 — блондином, 0.1 — рыжим. Выбирается группа из шести человек. Найти вероятности следующих событий:

\begin{enumerate}
    \item Событие $A$ — хотя бы один рыжий
    \item Событие $B$ — в составе группы не меньше четырех блондинов
    \item Событие $C$ — в составе группы равное число блондинов и шатенов
\end{enumerate}

\textbf{Решение:}

\begin{enumerate}
    \item Вероятность того, что есть хотя бы один рыжий:

    $P(A) = 1 - 0.9^6 \approx 0.468$
    \item Вероятность того, что не меньше четырех блондинов:

    $P(B) = C_{6}^{4} 0.4^{4} 0.6^{6 - 4} + C_{6}^{5} 0.4^{5} 0.6^{6 - 5} + C_{6}^{6} 0.4^{6} 0.6^{6 - 6} = 0.1792$
    \item Вероятность того, что равное число блондинов и шатенов:
    
    $P(C) = P(C_{00}) + P(C_{11}) + P(C_{22}) + P(C_{33})$

    $P(C_{00}) = (1-0.7)^{6} \approx 0.0007$

    Вероятность того, что будет один блондин, один шатен и четыре остальных:

    $P(C_{11}) = P_6(1; 1; 4) = \frac{6!}{1!1!4!} * 0.3 * 0.4 * (1 - 0.7)^4 \approx 0.03$

    Вероятность того, что будет два блондина, два шатена и два остальных:

    $P(C_{22}) = P_6(2; 2; 2) = \frac{6!}{2!2!2!} * 0.3 * 0.4 * (1 - 0.7)^2 \approx 0.1215$

    Вероятность того, что будет три блондина, три шатена и ноль остальных:

    $P(C_{33}) = P_6(3; 3) = \frac{6!}{3!3!} * 0.3 * 0.4 \approx 0.03$

    $P(C) = 0.0007 + 0.03 + 0.1215 + 0.03 \approx 0.18$
\end{enumerate}

\paragraph{Задача №2}

Производится стрельба по цели тремя снарядами. Снаряды попадают в цель независимо друг от друга. Для каждого снаряда вероятность попадания в цель $0.4$. Если в цель попал один снаряд, он ее поражает с вероятностью $0.3$. Если в цель попало два снаряда — с вероятностью $0.7$. Если в цель попало три снаряда — с вероятностью $0.9$. \textbf{Найти полную вероятность поражения цели}.

Гипотезы:

\begin{enumerate}
    \item $H_{1}$ — попал один снаряд
    \item $H_{2}$ — попало два снаряда
    \item $H_{3}$ — попало три снаряда
\end{enumerate}

Найдем вероятности для каждой из них:

\begin{enumerate}
    \item $P(H_{1}) = C_{3}^{1} * 0.4 * 0.6^{2} = 0.432$
    \item $P(H_{2}) = C_{3}^{2} * 0.4^{2} * 0.6 = 0.288$
    \item $P(H_{3}) = 0.4^3 = 0.064$
\end{enumerate}

\textbf{Ответ}:

$P(A) = 0.432 * 0.3 + 0.288 * 0.7 + 0.064 * 0.9 = 0.3888$

\paragraph{Задача №3}

Имеется $n$ лунок, по которым случайным образом разбрасывается $m$ шариков. Найти вероятность того, что в первую лунку попадет ровно $k$ шариков.

\textbf{Ответ:}

$P(A) = C_{m}^{k} * (\frac{1}{n})^{k} * (1 - \frac{1}{n})^{m - k}$

\paragraph{Задача №4}

Мишень состоит из яблока и двух колец. При одном выстреле вероятность попадания в яблоко равняется 0.4, вероятность попадания в первое кольцо — 0.5, вероятность попадания во второе кольцо — 0.6, вероятность непопадания — 0.3.

По мишени производится пять выстрелов. \textbf{Найти вероятность того, что они дадут два попадания в яблоко и одно попадание во второе кольцо}.

\textbf{Ответ:}

$P(A) = P_{5}(2;0;1;2) = \frac{5!}{2!*0!*1!*2!} * 0.4^2 * 0.5^0 * 0.6^1 * 0.3^2 = 0.2592$

\paragraph{Задача №5}

Производится стрельба пятью снарядами по группе, состоящей из трех целей. Обстрел ведется в следующем порядке: сначала обстреливается первая цель и огонь по ней ведется до тех пор, пока она или не будет поражена, или не кончатся все пять снарядов. После поражения обстрел переходит на следующую цель и так далее. Вероятность поражения цели при стрельбе по ней одним выстрелом — $p$

Найти вероятности того, что будет поражено:

\begin{enumerate}
    \item 0 целей
    \item 1 цель
    \item 2 цели
    \item 3 цели
\end{enumerate}

Решение:

\begin{enumerate}
    \item $P_0 = (1 - p)^5$
    \item $P_1 = P_{5}(1) = C_{5}^{1} p (1 - p)^4$
    \item $P_2 = P_{5}(2) = C_{5}^{2} p^2 (1 - p)^3$
    \item $P_3 = P_{5}(3) = C_{5}^{3} p^3 (1 - p)^2$
\end{enumerate}

\paragraph{Задача №6}

Прибор, состоящий из четырех узлов, работал в течение времени $t$. Надежность (вероятность безотказной работы) каждого узла за это время $t$ равна $0.7$. По истечении времени $t$ прибор останавливается, техник осматривает его и заменяет узлы, вышедшие из строя. На замену одного узла ему требуется время $\tau$. Найти вероятность того, что через время $2 \tau$ после остановки прибор будет готов для нормальной работы.

\textbf{Ответ:}

$P = P_{4}^{0} + P_{4}^{1} + P_{4}^{2} = C_{4}^{0} * 0.3^{0} * 0.7^{4} + C_{4}^{1} * 0.3^{1} * 0.7^{3} + C_{4}^{2} * 0.3^{2} * 0.7^{2} = 0.9163$

\paragraph{Задача №7}

В урне имеется $k$ шаров. Каждый из них с вероятностью $\frac{1}{2}$ может оказаться белым или черным. Из урны вынимается $n$ раз по одному шару. Причем каждый вынутый шар каждый раз возвращается обратно и шары перемешиваются. Среди вынутых $n$ шаров $m$ шаров оказались белыми. \textbf{Найти вероятность того, что из $k$ шаров в урне ровно $l$ белых}.

\textbf{Формула Байеса:}

$P(A|B) = \frac{P(B|A) P(A)}{P(B)}$

где

\begin{enumerate}
    \item $P(A)$ — априорная вероятность гипотезы $A$
    \item $P(A|B)$ — вероятность гипотезы $A$ при наступлении события $B$
    \item $P(B|A)$ — вероятность наступления события $B$ при истинности гипотезы $A$
    \item $P(B)$ — полная вероятность наступления события $B$
\end{enumerate}

\textbf{Решение:}

$P(H_{l}) = C_{k}^{l} (\frac{1}{2})^{k}$

$P(A|H_{l}) = C_{n}^{m} (\frac{l}{k})^{m} (1 - \frac{l}{k})^{n - m}$

$P(H_{l}|A) = \frac{P(A|H_{l}) P(H_{l})}{P(A)}$

\section{Практическое занятие — 27.10.2023}

\subsection{Случайные величины}

Случайные величины разделяются на

\begin{enumerate}
    \item \textbf{Дискретные} — конечное количество; например, цифры на кубике.
    \item \textbf{Непрерывные} — те случайные величины, которые разделяют сплошь промежуток, их может быть бесконечно много.
\end{enumerate}

\subsubsection{Дискретные случайные величины (ДСВ)}

\textbf{Законом распределения дискретной случайной величины} называют соответствие между полученными значениями дискретной случайной величины и их вероятностями.

Его можно задать:

\begin{enumerate}
    \item таблично;
    \item графически;
    \item аналитически (в виде формулы).
\end{enumerate}

\textbf{Ограничение}:

$$
\sum\limits_{i = 1}^{n} p_{i} = 1
$$

\textbf{Функция распределения} обозначается следующим образом: $F_{X}(x)$, $F(x)$ — вероятность того, что случайная величина примет значение \textbf{строго меньше} $x$.

\textbf{Свойства} функции распределения:

\begin{multicols}{2}
\begin{enumerate}
    \item $0 \le F(x) \le 1$
    \item $\lim\limits_{x \to -\infty} F(x) = 0$
    \item $\lim\limits_{x \to +\infty} F(x) = 1$
    \item $F(x_2) \ge F(x_1)$, если $x_2 \ge x_1$
\end{enumerate}
\end{multicols}

\paragraph{Числовые характеристики:}

\begin{enumerate}
    \item \textbf{Математическое ожидание}: $M(x)$, $m_x$, $E(x)$ — характеризует среднее арифметическое.
    
    $$M(x) = x_1p_1 + x_2 p_2 + \dots + x_{n} p_{n} = \sum\limits_{i = 1}^{n} x_{i} p_{i}$$

    $$M(x^2) = \sum\limits_{i = 1}^{n} x_{i}^2 p_{i}$$

    \begin{enumerate}
        \item Матожидание от любой константы равно константе: $M(c) = c$
        \item Константы выносятся за знак матожидания: $M(cx) = c M(x)$
        \item Матожидание суммы равно сумме матожиданий: $M(x + y) = M(x) + M(y)$
        \item Матожидание произведения равно произведению матожиданий: $M(x * y) = M(x) * M(y)$
    \end{enumerate}

    \textbf{Отклонение} от математического ожидания:

    $$X - M(x)$$
    \item \textbf{Дисперсия}: $D(x)$ — матожидание от отклонения в квадрате.
    
    $$D(x) = M((x - M(x))^2)$$

    \begin{multicols}{2}
    \begin{enumerate}
        \item $D(c) = 0$
        \item $D(cx) = c^2 D(x)$
        \item $D(x + y) = D(x) + D(y)$
        \item $D(x - y) = D(x) + D(y)$
    \end{enumerate}
    \end{multicols}
    \item \textbf{Среднеквадратическое отклонение}:
    
    $\sigma(x) = \sqrt{D(x)}$
\end{enumerate}

\begin{theorem}
    Математическое ожидание независимых случайных величин в $n$ испытаниях при постоянной и одинаковой вероятности успеха в каждом испытании $p$:

    $$M(x) = np$$
\end{theorem}

\begin{theorem}
    Матожидание отклонения случайной величины от математического ожидания равно нулю:

    $$M(x - M(x)) = 0$$

    \textbf{Доказательство}:

    $M(X - M(x)) = M(x) - M(M(x)) = M(x) - M(x) = 0$
\end{theorem}

\begin{theorem}
    Дисперсия в $n$ независимых испытаниях при постоянной и одинаковой вероятности успеха в каждом испытании $p$:

    $$D(x) = np(1-p)$$
\end{theorem}

\begin{theorem}
    Дисперсия случайной величины $x$:
    $$D(x) = M(x^2) - M(x)^2$$

    \textbf{Доказательство}:

    $D(x) = M(x - M(x))^2 = M(x^2 - 2xM(x) + M(x)^2) = M(x^2) 2 M(x) M(M(x)) + M(M(x)^2) = M(x^2) - M(x)^2$
\end{theorem}

\begin{definition}
    Случайная величина $x_0$ называется \textbf{нормированной}, если
    
    \begin{enumerate}
        \item $M(x_0) = 0$
        \item $\sigma(x_0) = 1$
    \end{enumerate}

    Если $M(x) = a$, $\sigma(x) = \sigma$, то $$x_{0} = \frac{x - a}{\sigma}$$

    Доказательство:
    
    \begin{enumerate}
        \item $M(x_0) = \frac{1}{\sigma} M(x - a) = \frac{1}{\sigma} (M(x) - M(a)) = \frac{1}{\sigma} (a - a) = 0$
        \item $D(x_0) = \frac{1}{\sigma^2} D(x - a) = \frac{1}{\sigma^2} (D(x) + D(a)) = \frac{1}{\sigma^2} * \sigma^2 = 1$
    \end{enumerate}
\end{definition}

\begin{definition}
    \textbf{Модой дискретной случайной величины} $x$ называют значение, которое принимается с наибольшей вероятностью. Обозначение: $m_o x$

    $$max \ P(x = x_i) = P(x = m_o x)$$
\end{definition}

\begin{definition}
    \textbf{Медианой дискретной случайной величины} $x$ является такое число, для которого вероятность меньших значений меньше $0.5$ и вероятность больших значений меньше $0.5$. Обозначение: $m_{e} x$
\end{definition}


\paragraph{Пример №1}

Закон распределения для шестигранного кубика.

$$
\begin{pmatrix}
  1 & 2 & 3 & 4 & 5 & 6 \\
  \frac{1}{6} & \frac{1}{6} & \frac{1}{6} & \frac{1}{6} & \frac{1}{6} & \frac{1}{6}
\end{pmatrix}
$$

\textbf{Функция распределения}:

$$F(x) = \begin{cases}
    0, \ x \le 1 \\
    \frac{1}{6}, 1 < x \le 2 \\
    \frac{2}{6}, 2 < x \le 3 \\
    \dots \\
    1, 6 < x
\end{cases}
$$

По ней мы можем нарисовать простенький график, что оставляется в качестве упражнения читателю.

Также найдем \textbf{математическое ожидание}: $M(x) = \frac{1}{6} (1 + 2 + 3 + 4 + 5 + 6) = \frac{21}{6}$

$D(x) = \frac{1}{6} ((-2.5)^2 + (-1.5)^2 + (-0.5)^2 + (0.5)^2 + (1.5)^2 + (2.5)^2) = \frac{35}{12} \approx 2.92$

\paragraph{Пример №2}

Найдем математическое ожидание для следующих величин:

$$
\begin{pmatrix}
    x & 0.01 & -0.01 \\
    p & \frac{1}{2} & \frac{1}{2}
\end{pmatrix}
$$


$$
\begin{pmatrix}
    y & 100 & -100 \\
    p & \frac{1}{2} & \frac{1}{2}
\end{pmatrix}
$$

Матожидание для обеих этих величин — ноль.

\paragraph{Пример №3}

Производится испытание на надежность трех видеорегистраторов. Вероятность то, что видеорегистратор выйдет из строя: $0.15$. Найти математическое ожидание и дисперсию.

\textbf{Закон распределения}:

$$
\begin{pmatrix}
    x & 0 & 1 & 2 & 3 \\
    p & 0.15^3 & & & 0.85^3
\end{pmatrix}
$$

$P_3(1) = C_{3}^{1} 0.15^2 * 0.85 \approx 0.0574$

$P_3(2) = 3 * 0.15 * 0.85^2 \approx 0.3251$

$$
\begin{pmatrix}
    x & 0 & 1 & 2 & 3 \\
    p & 0.0034 & 0.0574 & 0.3251 & 0.6141
\end{pmatrix}
$$

\textbf{Математическое ожидание}: $$M(x) = 1 * 0.0574 + 2 * 0.3251 + 3 * 0.6141 \approx 2.55$$

\textbf{Дисперсия}:

$D(x) = M(x^2) - (2.55)^2 = 0.3825$

Могли бы посчитать математическое ожидание и дисперсию с помощью формул:

$$
M(x) = 3 * 0.85 \approx 2.55
$$

$$
D(x) = 3 * 0.85 * 0.15 \approx 0.3825
$$

\paragraph{Пример №4}

$$
\begin{pmatrix}
    x & 0 & 1 \\
    p & 0.8 & 0.2
\end{pmatrix}
$$

\textbf{Найти}:

\begin{enumerate}
    \item $m_{o} x = 0$;
    \item $m_{e} x$ не существует.
\end{enumerate}

\section{Лекция — 28.10.2023}

\subsection{Дискретные случайные величины}

\subsubsection{Биномиальное распределение}

$P_n(k) = C_n^K p^k q^{n - k}$

\textbf{Закон биномиального распределения}:

$$
\begin{pmatrix}
  x & 0 & 1 & \dots & k & & n \\
  p & q^{n} & n p q^{n - 1} & & C_n^k p^k q^{n - k} & & p^n
\end{pmatrix}
$$

$$
F(x) = \begin{cases}
    0, \ x \le 0 \\
    \sum\limits_{k} C_{n}^{k} p^{k} q^{n - k}, \ 0 < x \le n \\
    1, \ x > n
\end{cases}
$$

\textbf{Параметры}: $n$, $p$

\textbf{Числовые характеристики}:

\begin{enumerate}
    \item \textbf{Математическое ожидание}:
    $M(x) = np$
    \item \textbf{Дисперсия}:
    $D(x) = n(1-p)p$
\end{enumerate}

Если $np \in Z$, то это будет \textbf{наивероятнейшее число успешных испытаний}.

\subsubsection{Пуассоновское распределение}

Если $n \to \infty$, $p \to 0$, $P_n(k) = \frac{\lambda^k}{k!} e^{-\lambda}$, где $\lambda = np$.

\textbf{Закон Пуассоновского распределения}:

$$
\begin{pmatrix}
  x & 0 & 1 & \dots & k & & n \\
  p & e^{-\lambda} & \lambda e^{-\lambda} & & \frac{\lambda^{k}}{k!} e^{-\lambda} & & \frac{\lambda^{n}}{n!} e^{-\lambda}
\end{pmatrix}
$$

\textbf{Параметры}: $\lambda$

\textbf{Числовые характеристики}:

\begin{enumerate}
    \item \textbf{Математическое ожидание}:
    $M(x) = \sum\limits_{k = 0}^{\infty} k \frac{\lambda^{k}}{k!} e^{-\lambda} = \sum\limits_{k = 1}^{\infty} k \frac{\lambda^{k}}{k!} e^{-\lambda} = \lambda e^{-\lambda} \sum\limits_{k = 1}^{\infty} \frac{\lambda^{k - 1}}{(k - 1)!} = \lambda e^{-\lambda} \sum\limits_{k = 0}^{\infty} \frac{\lambda^{k}}{k!} = \lambda e^{-\lambda} e^{\lambda} = \lambda$
    \item \textbf{Дисперсия}:
    $D(x) = M(x^2) - (M(x)^2)=\dots=\lambda$
\end{enumerate}

\subsubsection{Геометрическое распределение}

Рассматривается серия $n$ независимых испытаний, в которых успех появляется с одинаковой вероятностью $p$ и эти испытания \textbf{заканчиваются как только наступает успех}: $P_{k} = (1-p)p^{k - 1}$

\textbf{Закон геометрического распределения}:

$$
\begin{pmatrix}
    x & 1 & 2 & \dots & k \\
    p & p & (1-p)p & \dots & (1-p)^{k - 1}p
\end{pmatrix}
$$

\textbf{Параметры}: $p$

\textbf{Числовые характеристики}:

\begin{enumerate}
    \item \textbf{Математическое ожидание}: $M(x) = \frac{1}{p}$
    \item \textbf{Дисперсия}: $D(x) = \frac{1-p}{p^2}$
\end{enumerate}

\subsubsection{Решение задач}

\paragraph{Пример №1}

Вероятность того, что электроприбор откажет — 0.15. Сколько часов в среднем прибор отрабатывает до первого сбоя?

\textbf{Ответ}: $M(X) = \frac{1}{0.15} \approx 6.67$ (часов)

\subsection{Непрерывные случайные величины}

\begin{definition}
    \textbf{Случайная величина $X$ называется непрерывной}, если её функция распределения $F(x)$ является непрерывной.
\end{definition}

\begin{definition}
    \textbf{Вероятность того, что случайная величина $X$ примет значение из промежутка} $[a; b)$ равняется приращению функции распределения на этом промежутке:

    $$
        P(a \le X < b) = F(b) - F(a)
    $$
\end{definition}

\subsubsection{Плотность распределения}

\begin{definition}
    \textbf{Плотностью распределения вероятностей непрерывной случайной величины} называется первая производная её функции распределения, то есть

    $$f(x) = F'(x)$$
\end{definition}

\paragraph{Свойства плотности распределения}

\begin{enumerate}
    \item \textbf{Плотность распределения неотрицательная функция}, то есть
    
    $$\forall x \in (-\infty; \infty) f(x) > 0$$
    \item \textbf{Вероятность того, что случайная величина примет какое-либо значение, равняется единице}: 
    
    $$\int\limits_{-\infty}^{\infty} f(x) \diff x = 1$$
    \item $\lim\limits_{x \to \pm \infty} f(x) = 0$
\end{enumerate}

\paragraph{Пример №1}

Пусть $f(x) = c \arctg x$. При каком $c$ $f(x)$ будет являться плотностью распределения? Рассмотрим $\lim\limits_{x \to \pm \infty} c \arctg x = \pm \frac{c \pi}{2}$. Чтобы это выражение было равно нулю, $c$ должно быть равно нулю. Но интеграл единице ни при каком $c$ не будет равняться. \textbf{Следовательно, $f(x)$ ни при каком $c$ не будет плотностью распределения}.

\subsubsection{Функция распределения}

\begin{multicols}{2}
    \begin{enumerate}
        \item $F'(x) = f(x)$
        \item $F(x) = \int\limits_{\infty}^{x} f(t) \diff t$
        \item $\lim\limits_{x \to -\infty} F(x) = 0$
        \item $\lim\limits_{x \to +\infty} F(x) = 1$
    \end{enumerate}        
\end{multicols}

Из этих свойств следует:

$$
P(\alpha < x < \beta) = F(\beta) - F(\alpha)
$$

\paragraph{Пример №1}

Найти плотность распредления непрерывной случайной величины, если

$$
F(x) = \begin{cases}
    0, \ x \le 0 \\
    x^2, \ x \in (0; 1] \\
    1, \ x > 1
\end{cases}
$$

Просто, по определению, найдем производные:

$$
f(x) = \begin{cases}
  0, \ x \le 0 \\
  2x, \ x \in (0; 1] \\
  0, x > 1  
\end{cases}
$$

\subsubsection{Числовые характеристики}

\begin{enumerate}
    \item \textbf{Математическое ожидание}: $M(x) = \int\limits_{-\infty}^{+\infty} x f(x) \diff x$
    \item \textbf{Дисперсия}: $D(x) = \int\limits_{-\infty}^{+\infty} (x - M(x))^2 f(x) \diff x$. Но не забываем про то, что мы все так же можем считать по формуле: $D(x) = M(x^2) - M(x)^2$, где $M(x^2) = \int\limits_{-\infty}^{\infty} x^2 f(x) \diff x$
    \item \textbf{Среднеквадратическое отклонение}: $\sigma(x) = \sqrt{D(x)}$
    \item \textbf{Модой непрерывной величины} называют то ее значение, которому соответствует максимальное значение функции плотности.
    \item \textbf{Медианой непрерывной величины} называется её значение, при котором имеет место равенство
    
    $$P(X < M_{e}) = P(X > M_{e})$$

    \textbf{Оптимальное свойство медианы}. Сумма произведений отклонений значений случайной величины от медианы на соответствующие вероятности будет меньше, чем от любой другой величины:

    $$\sum\limits_{i = 1}^{n} |x_{i} - M_{e}| * p_{i} = min$$
    \item \textbf{Начальный момент}: $M(x^{k}) = \int\limits_{-\infty}^{\infty} x^{k} f(x) \diff x$
    
    Для дискретной случайной величины: $\alpha^{k} = M(x^{k}) = \sum\limits_{k = 1}^{n} x^{k} P_{k}$
    \item \textbf{Цетральный момент}: $\mu_k = M(x - M(x))^{k} = \int\limits_{-\infty}^{+\infty} (x - M(x))^{k} f(x) \diff x$
    \item \textbf{Коэффициент ассиметрии}: $A = \frac{\mu_{3}}{\sigma^3}$. Характеризует асимметрию распределения данной случайной величины. Неформально говоря, коэффициент асимметрии положителен, если правый хвост распределения длиннее левого, и отрицателен в противном случае.
    \item \textbf{Эксцесс}: $E = \frac{\mu^4}{\sigma^4} = -3$ служит для сравнения данного распределения с нормальным распределением. Если эксцесс у распределения положителен, то кривая будет более островершинной. Если эксцесс распределения отрицателен, то пик будет гладким.
\end{enumerate}

\textbf{Если один из интегралов расходится, то этой случайной величины не существует}.

\subsubsection{Решение задач}

\paragraph{Пример №1}

Пусть $$f(x) = \begin{cases}
    0, \ x < -2 \\
    -\frac{x^{3}}{4}, \ -2 \le x \le 0 \\
    0, \ x > 0
\end{cases}$$

Найти \textbf{коэффициент ассиметрии} и \textbf{эксцесс}.

\begin{enumerate}
    \item \textbf{Коэффициент ассиметрии}. Найдем математическое ожидание:
    
    $\alpha_1 = M(x) = \int\limits_{-2}^{0} x (-\frac{x^3}{4}) \diff x = - \frac{x^5}{20} \bigg|_{-2}^{0} = - \frac{32}{20} \approx - 1.6$. 
    
    $\alpha_2 = M(x^2) = -\frac{x^6}{24} \bigg|_{-2}^{0} = \frac{64}{24} \approx 2.67$, $\mu_2 = \alpha^2 - \alpha_1^2 = 2.67 - (1.6)^2 \approx 0.11$, $\alpha_3 = -4.57$, $\mu_3=\alpha_3 - 3\alpha_1 \alpha_2 + 2 \alpha_1^3 = 0.05$, $\alpha_4 = 8$, $\mu_4 = \alpha_4 - 4\alpha_1 \alpha_3 + 6\alpha_1^2 \alpha_2 - 3\alpha_1^4 \approx 0.1$

    Теперь можем посчитать коэффициент ассиметрии: $A = \frac{0.05}{\sqrt{0.11}^3} = \frac{0.05}{0.33^3} \approx 1.39$
    \item \textbf{Эксцесс}. Воспользуемся уже посчитанными в ходе предыдущих вычислений. Получается: $E = \frac{\mu^{4}}{\sigma^4} - 3 = \frac{0.1}{0.33^4} - 3 \approx 5.4$
\end{enumerate}

\section{Практическое занятие — 28.10.2023}

\subsection{Непрерывные случайные величины}

\subsubsection{Решение задач}

\paragraph{Пример №1} В стране пять крупных автогигантов. В кризисных условиях риск того, что завод обанкротится, составляет 30\%. Написать закон распределения автогигантов, которые могут обанкротиться в предстоящий кризисный период. Найти математическое ожидание и дисперсию.

Используем биномиальное распределение, $x$ — количество возможно обанкротившихся заводов:

$$
\begin{pmatrix}
  x & 0 & 1 & 2 & 3 & 4 & 5 \\
  p & 0.7^5 & 5*0.3*0.7^4 & 10*0.3^2*0.7^3 & 10*0.3^3*0.7^2 & 10*0.3^4*0.7 & 0.3^5  \\
    & 0.168 & 0.36 & 0.31 & 0.132 & 0.028 & 0.002
\end{pmatrix}
$$

\begin{enumerate}
    \item $M(x) = 5 * 0.3 = 1.5$
    \item $D(x) = 1.05$
\end{enumerate}

Таким образом, ожидать можно, что штуки две автогигантов рухнут, а чтобы все рухнули — можно не ожидать, \textbf{всё нормально}.

\paragraph{Пример №2} Количество ДТП в городе за неделю является случайной величиной, распределенной согласно Пуассоновскому распределению со средним значением, равным трём. \textbf{Какова вероятность того, что случится меньше трех ДТП}?

$M(x) = \lambda = 3$

$P(x < 3) = P(0) + P(1) + P(2) = e^{-3} (1 + 3 + \frac{9}{2}) \approx 0.42$

\textbf{Ответ:} 0.42

\paragraph{Пример №3}

Пусть $$F(x) = \begin{cases}
    0, \ x \le 1 \\
    (x - 1)^2, \ 1 < x \le 2 \\
    1, \ x > 2
\end{cases}
$$

Построить функцию распределения и найти вероятность того, что СВ примет значение

\begin{multicols}{2}
    \begin{enumerate}
        \item Принадлежащее $(1.2; 1.6)$
        \item Принадлежашее $[1.7; 2.3]$
        \item Принадлежащее $\{ x: x > 1.5 \}$
        \item Принадлежащее $\{ x, \ x \le 1.3 \}$
    \end{enumerate}    
\end{multicols}

Решение:

\begin{enumerate}
    \item $P(1.2 < x < 1.6) = F(1.6) - F(1.2) = 0.32$
    \item \dots
    \item $P(1.5 < x < +\infty) = 1 - F(1.5) = 1 - 0.25 = 0.75$
    \item $P(1.3 < x < +\infty) = F(1.3) = 0.09$
\end{enumerate}

\paragraph{Пример №4}

Пусть $f(x) = \frac{c}{1 + 9x^2}$. Найти значение $c$.

$\int\limits_{-\infty}^{+\infty} \frac{c}{1 + 9x^2} \diff x = 1$

$\int\limits_{-\infty}^{+\infty} \frac{c}{1 + 9x^2} = \lim\limits_{a \to -\infty, b \to +\infty} \int\limits_{a}^{b} \frac{c \diff x}{1 + 9x^2} = \lim\limits_{a \to -\infty, b \to +\infty} \int\limits_{a}^{b} \frac{c}{3} \arctg 3x \bigg|_{a}^{b} = \lim\limits_{a \to -\infty, b \to +\infty} \frac{c}{3} (\arctg 3b - \arctg 3a) = \frac{c}{3} (\frac{\pi}{2} + \frac{\pi}{2}) = \frac{\pi c}{3} = 1 \implies c = \frac{3}{\pi}$

\textbf{Ответ}: $\frac{3}{\pi}$

\paragraph{Пример №5}

Пусть $$f(x) = \begin{cases}
    0, \ x \le 2 \\
    2x - 4, \ 2 < x \le 3 \\
    0, \ x > 3
\end{cases}$$

Найти $F(x)$:

\begin{enumerate}
    \item Если $x \in (-\infty; 2]$
    
    $F(x) = \int\limits_{-\infty}^{2} 0 \diff t = 0$
    \item Если $x \in (2; 3]$
    
    $F(x) = \int\limits_{2}^{x} (2x - 4) \diff x = (x^2 - 4x) \bigg|_{2}^{x} = x^2 - 4x + 4 = (x-2)^2$
    \item Если $x \in (3; +\infty)$
    
    $F(x) = 1$
\end{enumerate}

\paragraph{Пример №6}

Пусть $$f(x) = \begin{cases}
    0, x \le 0 \\
    2e^{-2x}, \ x > 0 
\end{cases}$$

Определить \textbf{математическое ожидание}. $M(x) = \int\limits_{-\infty}^{+\infty} x f(x) \diff x = \int\limits_{-\infty}^{0} x 0 \diff x + \int\limits_{0}^{+\infty} x 2 e^{-2x} \diff x = \int\limits_{0}^{+\infty} x e^{-2x} \diff x = \lim\limits_{b \to +\infty} 2 \int\limits_{0}^{b} x e^{-2x} \diff x = \begin{vmatrix}
    u = x & \diff u = \diff x \\
    \diff v = 2 e^{-2} \diff x & v = -e^{-2x}
\end{vmatrix} = \lim\limits_{b \to +\infty} (-x e^{- 2x} \bigg|_0^b + \int\limits_{0}^{b} e^{-2x} \diff x) = \lim\limits_{b \to +\infty} (-b e^{- 2 b} - \frac{1}{2} e^{-2x} \bigg|_{0}^{b}) = \lim\limits_{b \to +\infty} (-b e^{-2 b} - \frac{1}{2} e^{-2b} + \frac{1}{2}) = 0.5$

\paragraph{Пример №7}

Пусть $$f(x) = \begin{cases}
    0, \ x \le 1 \\
    -6x^2 + 18x - 12, \ 1 < x < 2 \\
    0, x \ge 2
\end{cases}$$

\textbf{Найти}: $M_e x$

Для того, чтобы найти $M_e x$, необходимо найти $F(x)$.

$\int\limits_{1}^{x} (-6x^2 + 18x - 12) \diff x = -2x^3 + 9x^2 - 12 x \bigg|_1^x = -2x^3 + 9x^2 - 12x + 5$

$$F(x) = \begin{cases}
    0, \ x \le 1 \\
    -2x^3 + 9x^2 - 12x + 5, \ 1 < x < 2 \\
    1, \ x \ge 2
\end{cases}$$

$$-2x^3 + 9x^2 - 12x + 5 = \frac{1}{2}$$

$$x_1 = \frac{3}{2}, x_2 = \frac{3}{2} - \frac{\sqrt{3}}{2}, x_3 = \frac{3}{2} + \frac{\sqrt{3}}{2}$$

\textbf{Ответ}: $\frac{3}{2}$

\subsection{Равномерное распределение}

Говорят, что СВ имеет непрерывное равномерное распределение на отрезке $[ a , b ]$ $[a,b]$, где $a,b \in R$, если её плотность $f_{X}(x)$ имеет вид: 

$$f(x) = \begin{cases}
    c, \ x \in (a, b) \\
    0, \ x \notin (a, b)
\end{cases} = \begin{cases}
    \frac{1}{b - a}, \ x \in (a, b) \\
    0, \ x \notin (a, b)
\end{cases}$$

\subsubsection{Функция распределения}

Интегрируя определённую выше плотность, получаем: 

$$F_x(x) = P(X \le x) = \begin{cases}
    0, \ x < a \\
    \frac{x - a}{b - a}, \ a \le x < b \\
    1, \ x \ge b
\end{cases}$$

\subsubsection{Числовые характеристики}

\begin{multicols}{2}
    \begin{enumerate}
        \item \textbf{Математическое ожидание:} $M(x) = \frac{a + b}{2}$
        \item \textbf{Дисперсия}: $D(x) = \frac{(b - a)^2}{12}$
        \item \textbf{Значение медианы}: $\frac{a + b}{2}$
        \item \textbf{Значение моды}: любое число из $[a, b]$
        \item \textbf{Коэффициент асимметрии}: $0$
        \item \textbf{Эксцесс}: $-\frac{6}{5}$
    \end{enumerate}    
\end{multicols}

\subsubsection{Решение задач}

\paragraph{Пример №1} В некотором городе трамвай ходит по расписанию с интервалом 10 минут. Найти вероятность того, что пассажир, подошедший к остановке, будет ожидать трамвай более 2 минут.

\textbf{Ответ:} $P = 0.8$

\paragraph{Пример №2} Обследуются животные, каждое из которых с вероятностью $p$ является больным. Обследование производится путем анализов крови. Если смешать кровь $n$ животных, то анализ этой смеси будет положительным, если среди этих $n$ животных хотя бы одно больное. Требуется обследовать большое число животных $N$. Предлагается несколько способов обследования:

\begin{enumerate}
    \item \textbf{Обследовать всех этих животных}, проведя $N$ анализов;
    \item \textbf{Вести обследование по группам}, смешав сначала кровь группы из $n$ животных, если все хорошо — перейти к следующей группе, если нет — сделать анализ каждого из животных.
\end{enumerate}

\textbf{Определить, какой способ обследования наиболее выгодный} (в плане минимального проведения анализов): первый или второй?

\textbf{Закон распределения для группы животных}:

$$
\begin{pmatrix}
    x & 1 & n + 1 \\
    p & (1-p)^{n} & 1 - (1-p)^{n} 
\end{pmatrix}
$$

$M(x_1) = n$

$M(x_2) = (1-p)^{n} + (n + 1)(1 - (1-p)^{n}) = (1-p)^{n} + n - n(1-p)^{n} + 1 - (1-p)^{n} = n - n(1-p)^{n} + 1$

Предполагаем, второй способ выгоднее: $n - n(1-p)^{n} + 1 > n \Longleftrightarrow n(1-p)^{n} < 1$ — \textbf{если это значение меньше единице, используем второй способ; если больше — первый способ}.

\subsection{Домашнее задание}

\paragraph{Задание №1}

Пусть $$F(x) = \begin{cases}
    0, \ x \le 1 \\
    (x - 1)^2, \ 1 < x \le 2 \\
    1, \ x > 2
\end{cases}
$$

Построить функцию распределения и найти вероятность того, что СВ примет значение

\begin{multicols}{2}
    \begin{enumerate}
        \item Принадлежашее $[1.7; 2.3]$
    \end{enumerate}    
\end{multicols}

\textbf{Решение (возможно, неправильное)}:

$P(1.7 \le x \le 2.3) = 1 - ((1 - F(2.3)) + F(1.7)) = 1 - (0 + 0.49) = 0.51$

\textbf{Ответ (возможно, неправильный)}: $0.51$

\paragraph{Задание №2}

Пусть

$$f(x) = \begin{cases}
    0, \ x \le 2 \\
    (x-2)^2, \ 2 < x \le 3 \\
    1, \ x > 3
\end{cases}$$

Найти $F(x)$

\textbf{Решение (возможно, неправильное)}:

$\int\limits^{x}_{2} ((x-2)^2) = (4 x - 2 x^2 + \frac{x^3}{3}) \bigg|^{x}_{2} = (4x - 2x^2 + \frac{x^3}{3}) - (8 - 8 + \frac{8}{3}) = 4x - 2x^2 + \frac{x^3}{3} - \frac{8}{3}$

\textbf{Ответ (возможно, неправильный)}:

$$F(x) = \begin{cases}
    0, \ x \le 2 \\
    4x - 2x^2 + \frac{x^3}{3}, \ 2 < x \le 3 \\
    1, \ x > 3
\end{cases}$$

\end{document}