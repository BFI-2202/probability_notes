\documentclass{article}
\usepackage[utf8]{inputenc}

\usepackage[T2A]{fontenc}
\usepackage[utf8]{inputenc}
\usepackage[russian]{babel}

\usepackage{tabularx}
\usepackage{amsmath}
\usepackage{pgfplots}
\usepackage{geometry}
\geometry{
    left=1cm,right=1cm,top=2cm,bottom=2cm
}
\newcommand*\diff{\mathop{}\!\mathrm{d}}

\newtheorem{definition}{Определение}
\newtheorem{theorem}{Теорема}

\DeclareMathOperator{\sign}{sign}

\usepackage{hyperref}
\hypersetup{
    colorlinks, citecolor=black, filecolor=black, linkcolor=black, urlcolor=black
}

\title{Теория вероятностей и математическая статистика}
\author{Лисид Лаконский}
\date{September 2023}

\begin{document}
\raggedright

\maketitle

\tableofcontents
\pagebreak

\section{Лекция — 30.09.2023}

\subsection{Повторение испытаний. Схема испытаний Бернули}

В серии $n$ независимых испытаний, в каждом из которых вероятность успеха $p$ является постоянной и одинаковой и вероятность неудачи $q$ тоже является постоянной, \textbf{вероятность того, что успех наступит ровно $k$ раз}, обозначается как $P_{n}(k)$ и вычисляется по формуле:

$$P_{n}(k) = C_{n}^{k} p^{k} q^{n - k}, \ 0 \le k \le n$$

\textbf{Вероятность того, что в $n$ испытаниях успех наступит от $k_{1}$ до $k_{2}$ раз}:

$$P_{n}(k_1, k_2) = \sum\limits_{k = k_1}^{k = k_2} C_{n}^{k} C_{n}^{k} p^{k} q^{n - k}$$

\textbf{Производящая функция}, вероятность наступления успеха $n$ раз: $q_{n}(x) = (q + px)^{n} = C_{n}^{0} q^{n} + C_{n}^{1} q^{n - 1} p x + C_{n}^{2} q^{n - 2} (p x)^{2} + \dots + C_{n}^{1} q^{n - k} (p x)^{k} + \dots + (p x)^{n}$

\textbf{Вероятность того, что в $n$ испытаний произойдет $m_{k}$ событий $A_{k}$ с вероятностью $p_{k}$} (нуждается в проверке)

$$P_{n}(m_1; m_2, \dots, m_k) = \frac{n!}{m_{1}! m_{2}! m_{k}!} * p_{1}^{m_1} p_{2}^{m_2} p_{k}^{m_k}$$

Не уследил, что это такое:

$$\frac{P_{n}(k + 1)}{P_{n}(k)} = \frac{C_{n}^{k + 1} p^{k + 1} q^{n - k - 1}}{C_{n}^{k} p^{k} q^{n - k}} = \frac{P(n - k)}{(k + 1) q} \ge 1$$

$$np - q \le k_0 \le np + p$$

\paragraph{Пример №1}

Вероятность того, что в течение суток расход электроэнергии не превысит нормы, равняется 0,9. Какова вероятность того, что в ближайшие 7 суток расход электроэнергии в течение четырех суток не будет превышать нормы.

\textbf{Решение}:

$P_{7}(4) = C_{7}^{4} * 0.9^{4} * 0.1^{3} = 35 * 0.9^{4} * 0.1^{3} = 0.023$

\textbf{Ответ}: 0.023

\paragraph{Пример №2}

Три камеры наблюдения работают независимо друг от друга. Вероятность зафиксировать событие первой камерой — 0.7, второй — 0.8, третьей — 0.9. Какова вероятность того, что событие будет зафиксировано:

\begin{enumerate}
    \item Тремя камерами
    \item Двумя камерами
    \item Одной камерой
    \item Не попадет в поле зрение ни одной камеры
\end{enumerate}

\textbf{Решим с помощью производящей функции}: $(p_1x + q_1) (p_2x + q_2)(p_3x + q_3) = \dots$. Все три камеры: суммы с коэффициентом $x^3$; две камеры: суммы с коэффициентом $x^2$ ;не попадет в поле зрение ни одной камеры: без $x$.

\textbf{Решение другим способом}:

$P(III) = 0.7 * 0.8 * 0.9 = 0.504$

$P(II) = p_1 p_2 q_3 + p_1 q_2 p_3 + q_1 p_2 p_3 = 0.398$

$P(0) = 0.3 * 0.2 * 0.1 = 0.006$

$P(I) = 1 - P(III) - P(II) - P(0) = 0.092$

\paragraph{Пример №3}

Проводится 21 испытание, $p = 0.4$. Найти наибольшую вероятность успеха в этих испытаниях.

\textbf{Используем известное неравенство}: $21 * 0.4 - 0.6 \le k_0 \le 21 * 0.4 + 0.4 \Leftrightarrow 7.8 \le k_0 \le 8.8$

\textbf{Ответ:} 8

\paragraph{Пример №4}

$p = 0.5$

$P_{8}(3, 5) = \sum\limits_{k = 3}^{5} C_{n}^{k} C_{n}^{k} p^{k} q^{n - k} = C_{8}^{3} (\frac{1}{2})^{3} (\frac{1}{2})^{5} + C_{8}^{4} (\frac{1}{2})^{8} + C_{8}^{5} (\frac{1}{2})^{8} = 0.71$

\textbf{Ответ:} 0.71

\paragraph{Пример №5}

Какая вероятность того, что в серии из 110 испытаний успех наступит 58 раз, если $p = 0.4$?

\textbf{Решение:}

$P_{110}(58) = C_{110}^{58} 0.4^{58} 0.6^{52} \approx 0.002025$

\textbf{Ответ:} 0.002025

\end{document}