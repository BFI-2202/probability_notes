\documentclass{article}
\usepackage[utf8]{inputenc}

\usepackage[T2A]{fontenc}
\usepackage[utf8]{inputenc}
\usepackage[russian]{babel}

\usepackage{tabularx}
\usepackage{amsmath}
\usepackage{pgfplots}
\usepackage{geometry}
\geometry{
    left=1cm,right=1cm,top=2cm,bottom=2cm
}
\newcommand*\diff{\mathop{}\!\mathrm{d}}

\newtheorem{definition}{Определение}
\newtheorem{theorem}{Теорема}

\DeclareMathOperator{\sign}{sign}

\usepackage{hyperref}
\hypersetup{
    colorlinks, citecolor=black, filecolor=black, linkcolor=black, urlcolor=black
}

\title{Теория вероятностей и математическая статистика}
\author{Лисид Лаконский}
\date{September 2023}

\begin{document}
\raggedright

\maketitle

\tableofcontents
\pagebreak

\section{Практическое занятие — 30.09.2023}

\subsection{Надёжность электрических схем}

См. \url{https://www.matburo.ru/tvart_sub.php?p=art_scheme}

\subsection{Решение задач}

\paragraph{Пример №1}

Рассматривается посадка вертолета на аэродром. Если позволяет погода, летчик сажает самолет, наблюдая на аэродром визуально, в этом случае вероятность благополучной посадки равна $p_1$. Если наблюдается облачность, то летчик сажает самолет по приборам, вероятность их благополучной работы равна $p_0$. Если приборы сработали хорошо, то вероятность благополучной посадки равна $p_1$. Если же приборы не сработали, то вероятность благополучной посадки равна $p^{*}$. \textbf{Найти полную вероятность благополучной посадки, если известно, что в $k$ процентах всех случаев посадки аэропорт затянут низкой облачностью}.

\textbf{Решение:}

$P(B_1) = \frac{k}{100}$ — облачность не наблюдается.

$P(B_2) = 1 - \frac{k}{100}$ — облачность не наблюдается.

$P(A)_{B_1} = p_0 p_1 + (1 - p_0) p^{*}$

$P(A)_{B_2} = p_1$

\paragraph{Пример №2}

В группе из 10 студентов, пришедших на экзамен, 3 подготовлены отлично, 4 — хорошо, 2 — посредственно, 1 — плохо. В экзаменационный билетах 20 вопросов, отлично подготовленный студент может ответить на 20 из них, хорошо — на 16 из них, посредственно — на 10 из них, плохо — на 5 из них.

Вызванный наугад студент ответил на 3 произвольно заданных вопроса. \textbf{Найти вероятность того, что этот студент подготовлен плохо}.

\textbf{Решение:}

Имеется четыре гипотезы:

\begin{enumerate}
    \item $B_{1}$ — студент подготовлен отлично
    \item $B_{2}$ — студент подготовлен хорошо
    \item $B_{3}$ — студент подготовлен посредственно
    \item $B_{4}$ — студент подготовлен плохо
\end{enumerate}

Найдем вероятности для каждой из них:

\begin{enumerate}
    \item $P(B_1) = \frac{3}{10}$
    \item $P(B_2) = \frac{2}{5}$
    \item $P(B_3) = \frac{1}{5}$
    \item $P(B_4) = \frac{1}{10}$
\end{enumerate}

Найдем условную вероятность для каждого из случаев:

\begin{enumerate}
    \item $P_{B_1}(A) = 1$
    \item $P_{B_2}(A) = \frac{16}{20} * \frac{15}{19} * \frac{14}{18} = 0.49122807017$
    \item $P_{B_3}(A) = \frac{10}{20} * \frac{9}{19} * \frac{8}{18} = 0.10526315789$
    \item $P_{B_4}(A) = \frac{5}{20} * \frac{4}{19} * \frac{3}{18} = 0.00877192982$
\end{enumerate}

$P(A) = \frac{3}{10} * 1 + \frac{2}{5} * 0.49 + \frac{1}{5} * 0.1 + \frac{1}{10} * 0.008 = 0.5168$

$P_{A}(B_4) = \frac{P(B_4) P_{B_4}(A)}{P(A)} = \frac{0.0008}{0.5163} \approx 0.002$

\textbf{Ответ:} 0.002

\paragraph{Пример №3} Цель, по которой ведется стрельба, состоит из двух различных по уязвимости частей. Для поражения цели достаточно одного попадания в первую часть или двух попаданий в вторую. Для каждого попавшего в цель снаряда вероятность попадания в первую часть равна $p_1$, а во вторую $p_2$, причем $p_2 = 1 - p_1$. По цели производится три выстрела. Вероятность попадания при каждом выстреле равна $p$. \textbf{Найти вероятность, что данными тремя выстрелами цель будет поражена}.

\textbf{Решение:}

Имеется три гипотезы:

\begin{enumerate}
    \item $B_1$ — в цель попал один снаряд;
    \item $B_2$ — в цель попало два снаряда;
    \item $B_3$ — в цель попало три снаряда.
\end{enumerate}

Найдем вероятности для каждой из них:

\begin{enumerate}
    \item $P(B_1) = 3 p (1 - p)(1 - p)$
    \item $P(B_2) = 3 p^2 (1 - p)$
    \item $P(B_3) = p^3$
\end{enumerate}

Найдем условную вероятность поражения цели для каждого из случаев:

\begin{enumerate}
    \item $P_{B_1}(A) = p_1$
    \item $P_{B_2}(A) = p_2^2 + (1 - (1 - p_1)^2)$
    \item $P_{B_3}(A) = 1$
\end{enumerate}

\textbf{Ответ:} $P(A) = 3p_1p (1-p)^2 + 3p^2 (1 - p) (p_2^2 + (1 - (1 - p_1)^2)) + p^3$

\paragraph{Пример №4}

Три орудия производят стрельбу по трем целям. Каждое орудие выбирает себе цель случайным образом и не зависимо от других. Цель, обстреленная одним орудием, поражается с вероятностью $p$. \textbf{Найти вероятность того, что из трех целей две будут поражены, а третья нет}.

Имеется три гипотезы:

\begin{enumerate}
    \item $B_1 = 1 \frac{1}{3} \frac{1}{3} = \frac{1}{9}$ — Все орудия стреляют по одной цели;
    \item $B_2 = 1 - \frac{1}{9} - \frac{2}{9} = \frac{2}{3}$ — Орудия стреляют по двум целям;
    \item $B_3 = \frac{2}{3} \frac{1}{3} = \frac{2}{9}$ — Орудия стреляют по трем целям.
\end{enumerate}

Найдем условную вероятность для каждого из случаев:

\begin{enumerate}
    \item $P_{B_1}(A) = 0$
    \item $P_{B_2}(A) = p (1 - (1 - p)^2)$
    \item $P_{B_3}(A) = 3p^2 (1 - p)$
\end{enumerate}

\textbf{Ответ:} $P(A) = \frac{2}{3} p (1 - (1 - p)^2) + \frac{2}{3} p^2 (1 - p)$

\subsection{Домашнее задание}

Три истребителя совершают налет на объект, который защищён четырьмя зенитными ракетами. Каждая ракета простреливает угловой сектор размерами 60°. Если истребитель пролетает через защищённый сектор, его обстреливают и поражают с вероятностью $p$. Через незащищённый сектор истребитель проходит без препятствий. Каждый истребитель, прошедший к объекту, сбрасывает бомбу и поражает объект с вероятностью $p_0$. Экипажи истребителей не знают, где расположены зенитные ракеты. \textbf{Найти вероятность поражения объекта для двух способов налета}:

\begin{enumerate}
    \item Все три истребителя летят по одному и тому же направлению, выбираемому случайно
    \item Каждый из самолетов выбирает себе направление случайно независимо от других
\end{enumerate}

\textbf{Подсказка}: берем две гипотезы

\begin{enumerate}
    \item Самолёты выбрали незащищённой направление
    \item Самолёты выбрали защищённое направление
\end{enumerate}

\textbf{Решение:}

\begin{enumerate}
    \item Все три истребителя летят по одному и тому же направлению, выбираемому случайно
    
    Имеем две гипотезы:

    \begin{enumerate}
        \item $B_1 = \frac{360 - 4 * 60}{360} = \frac{1}{3} $ — самолеты выбрали незащищенное направление
        \item $B_2 = 1 - B_1 = \frac{2}{3}$ — самолеты выбрали защищенное направление
    \end{enumerate}

    Найдем условную вероятность для каждого из случаев:

    \begin{enumerate}
        \item $P_{B_1}(A) = 1 - (1 - p_0)^3$
        \item $P_{B_2}(A) = 1 - (1 - (1 - p) p_0)^3$
    \end{enumerate}

    $P(A) = \frac{1}{3} (1 - (1 - p_0)^3) + \frac{2}{3} (1 - (1 - (1 - p) p_0)^3)$
    \item Каждый из самолетов выбирает себе направление случайно независимо от других

    Вероятность поразить объект для каждого из самолетов:

    $p_1 = \frac{1}{3} p_0 + \frac{1}{3} (1 - p) p_0$

    Вероятность, что хотя бы один из трех самолетов поразит объект:

    $P(A) = 1 - (1 - p_1)^3 = 1 - (1 - (\frac{1}{3} p_0 + \frac{1}{3} (1 - p) p_0))^3$
\end{enumerate}

\textbf{Ответ:}

\begin{enumerate}
    \item $P(A) = \frac{1}{3} (1 - (1 - p_0)^3) + \frac{2}{3} (1 - (1 - (1 - p) p_0)^3)$
    \item $P(A) = 1 - (1 - p_1)^3 = 1 - (1 - (\frac{1}{3} p_0 + \frac{1}{3} (1 - p) p_0))^3$
\end{enumerate}

\end{document}